\chapter{Método de Pesquisa}
\label{c.metodo}

A Figura 8 apresenta um diagrama com os principais elementos deste projeto de pesquisa. As elipses representam temas ou assuntos. Os retângulos com cantos arredondados, atividades. As setas representam as relações entre diferentes elementos. O cilindro representa uma base de dados de artigos científicos.

\begin{figure}[h]
	\caption{\small Modelo de Pesquisa.}
	\centering
	\includegraphics[scale=0.7]{Imagens/metodologia.png}
	\label{f.metodopesquisa}
	\legend{\small Fonte: Elaborada pelo autor.}
\end{figure}

A pesquisa será dividida em seis etapas. A princípio (fundamentação teórica) será feito um levantamento bibliográfico sobre os tipos de controle físicos que podem ser utilizados no celular, ferramentas para o desenvolvimento de aplicações em realidade virtual e métodos de avaliação para controles físicos.
 
A segunda etapa (preparação do ambiente operacional), envolve a escolha dos software a serem utilizadas com base na exequibilidade do projeto e da acessibilidade das ferramentas, ou seja, devem ser capazes de proporcionar as vias necessárias para o êxito do projeto preferencialmente de forma gratuita e com documentação clara. 

Na terceira fase do projeto (escolha dos controles físicos), será feita a escolha de três tipos de controles que apresentam três diferentes tipos de conexão e iteratividade: via cabo, Bluetooth e toque na tela. 

Na quarta etapa (desenvolvimento da aplicação em RV), será realizado o desenvolvimento da aplicação, se possível com o auxílio de um designer para um visual mais atrativo. Juntamente com o desenvolvimento, serão realizados os testes e correções da aplicação considerando a usabilidade da mesma.
 
A quinta etapa (análise dos controles), verificará se o projeto atingiu os objetivos geral e específicos propostos levando em consideração a análise dos controles em relação as suas características de usabilidade, experiência e funcionalidade. 

Na última (sexta) etapa (relatório de pesquisa), será elaborado o relatório final da pesquisa registrando todos os procedimentos realizados bem como os resultados e conclusões.


