\chapter{Análise dos Controles}
\label{c.analise}

A fim de avaliar tecnologias sob o aspecto de uso, \citeonline{mcnamara} propuseram uma estrutura que se baseia em três aspectos: funcionalidade, experiência e usabilidade. A funcionalidade leva em consideração as características técnicas do dispositivo, experiência foca no relacionamento entre o usuário e a tecnologia e usabilidade foca nas características de interação entre o usuário e o dispositivo. 


\section{Funcionalidade}
\label{funcionalidade}

Segundo \citeonline{mcnamara}, para avaliar a funcionalidade de um dispositivo pode-se analisar a performance, confiabilidade e durabilidade do mesmo. A quantidade de funções que um dispositivo oferece também deve ser considerada pois muitas opções de entrada podem inutilizar muitos comandos e poucas podem ser insuficientes. \citeonline{brown} afirma que a funcionalidade é um aspecto de interação que é relativamente independente do ambiente e do usuário. 

Quanto à performance, a característica a ser analisada será se o dispositivo oferece resposta rápida aos comandos do usuário, ou seja, se são observados atrasos na comunicação entre o dispositivo e o celular. A facilidade de conexão do controle ao dispositivo móvel e a preservação desta conexão serão características de confiabilidade. Por fim, a durabilidade levará em consideração o tipo de material de cada controle e se houve ou não falhas mecânicas na execução de comandos.

A análise foi feita com base em observações feitas pela autora onde cada controle foi utilizado para interagir com a aplicação desenvolvida. Cada controle foi analisado por em média X minutos (tempo médio para destruir todos os focos de dengue apresentados na aplicação) e as características de cada um podem ser observadas na tabela abaixo.

\begin{longtable}{p{2.8cm}|p{2.8cm}|p{2.8cm}|p{2.8cm}|p{2.8cm}}	
    \caption{Características de Funcionalidade} \\
    \label{t.funcionalidade} 
	\textbf{\small Características observadas } & \textbf{\small Google Cardboard 2.0} & \textbf{\small Controle PS2} & \textbf{\small Controle VRBox} & \textbf{\small Teclado } \\\hline
	
	{\small Ocorreram atrasos de resposta?} & {\small } & {\small } & {\small } & {\small } \\\hline	
		 
	{\small Dificuldade da conexão inicial} & {\small } & {\small } & {\small } & {\small } \\\hline		 
	
	{\small Houve problemas de conexão?} & {\small } & {\small } & {\small } & {\small } \\\hline  
			 
	{\small Qualidade do material} & {\small } & {\small } & {\small } & {\small } \\\hline
	
    {\small Houve falhas mecânicas?} & {\small } & {\small } &{\small } & {\small }\\\hline
	
\end{longtable}
	\legend{\small Fonte: Elaborada pelo autor}	
	
\section{Usabilidade}

A usabilidade, segundo a ISO 9241-11 \cite{iso9241} tem como objetivo definir usabilidade e explica como identificar a informação necessária para avaliação de usabilidade de um computador em termos de medidas de desempenho e satisfação do usuário, ou seja, mede o quanto um usuário específico pode utilizar um produto e atingir os seus objetivos com eficácia, eficiência e satisfação em um contexto específico de uso.

De acordo com \citeonline{brown}, eficácia descreve a habilidade do usuário em realizar uma tarefa com a tecnologia. Eficiência considera os recursos utilizados para realizar a tarefa, podem ser esforço mental, esforço físico ou tempo. A satisfação mede o quanto a interação impactou o usuário, devendo ser extraída somente através de respostas do usuário. 

Os questionários aplicados neste estudo tiveram como base a ISO 9241-9 e o trabalho de \cite{lewis} que é citado na ISO 9241-11 e oferece modelos de questionários para avaliação da satisfação de um produto. Por fim, o contexto de uso leva em consideração não somente o ambiente físico mas também as características individuais dos usuários. 

Para acessar a usabilidade oferecida por cada controle, X voluntários utilizaram a aplicação com os quatros diferentes controles e responderam a dois questionários elaborados por \cite{lewis}. O primeiro questionário é denominado "The After-Scenario Questionnaire (ASQ)" e é respondido para quatro cenários: andar, eliminar um foco de dengue, agachar e retornar ao menu. Este questionário possui apenas três questões que irão medir a facilidade para completar uma tarefa, tempo para completar a tarefa e se o suporte foi adequado. Como a análise é restrita aos controles e não à aplicação em si, o ASQ utilizado conterá duas questões já que problemas com o suporte não serão considerados. O segundo questionário, denominado "The Post-Study System Usability Questionnaire (PSSUQ)" é utilizado após a experiência completa da aplicação e será respondido para cada controle. Como no primeiro questionário, questões que levavam em consideração a aplicação foram removidas. Ambos os questionários utilizam uma escala de 1 a 7 para cada pergunta sendo que 1 corresponde à "concorda totalmente" e 7 "discorda totalmente". No final, é feito a média aritmética das notas onde menores médias representam melhores avaliações.  

No geral, X homens e X mulheres participaram da análise. A média de idade foi de X, com idades variando entre X e X anos. Também foi questionado se os participantes possuíam experiência com algum dos controles apresentados e se já haviam utilizado aplicações em RV. 

Os resultados do ASQ e do PSSUQ podem ser visualizados nas tabelas abaixo.

\begin{table}[H]	
	\caption{Resultados após cada cenário (ASQ)} 
	\label{t.ASQ} 
	\centering
	\begin{tabular}{l|l|l|l|l}
	\textbf{\small Dispositivos } & \textbf{\small Andar} & \textbf{\small Eliminar um foco de dengue} & \textbf{\small Agachar} & \textbf{\small Retornar ao menu} \\\hline
	
	{\small Google Cardboard 2.0} & {\small } & {\small } & {\small } & {\small } \\\hline	
	
	{\small Controle PS2} & {\small } & {\small } & {\small } & {\small } \\\hline		 
	
	{\small Controle VRBox} & {\small } & {\small } & {\small } & {\small } \\\hline  
	
	{\small Teclado} & {\small } & {\small } & {\small } & {\small } \\\hline	
	\end{tabular}
	\legend{\small Fonte: Elaborada pelo autor}	
\end{table}

\begin{table}[H]	
	\caption{Resultados após cada controle (PSSUQ)} 
	\label{t.ASQ} 
	\centering
	\begin{tabular}{l|l}
		\textbf{\small Dispositivos } & \textbf{\small Média}\\\hline
		
		{\small Google Cardboard 2.0} & {\small } \\\hline	
		
		{\small Controle PS2} & {\small } \\\hline		 
		
		{\small Controle VRBox} & {\small }  \\\hline  
		
		{\small Teclado} & {\small } \\\hline	
	\end{tabular}
	\legend{\small Fonte: Elaborada pelo autor}	
\end{table}

\section{Experiência}
\label{experiencia}

“(Experiência) remete à todas as qualidades do sistema interativo que o fazem memorável, satisfatório e gratificante.” \cite[tradução nossa]{benyon} Tendo em mente os controles de interação, usuários podem obter uma experiência negativa se as características de funcionalidades forem insatisfatórias e se houver problemas para encontrar os botões corretos no controle, já que neste caso poderá ser necessária a remoção do capacete de visualização resultando em uma interrupção da experiência em RV. 

Para verificar a experiência oferecida por cada controle, foi perguntado aos usuários qual dos quatro controles resultou em uma melhor experiência e comentários involuntários sobre cada controle foram anotados e separados entre positivos e negativos. 

