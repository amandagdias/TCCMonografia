\chapter{Análise dos Controles}
\label{c.analise}

A fim de avaliar tecnologias sob o aspecto de uso, \citeonline{mcnamara} propuseram uma estrutura que se baseia em três aspectos: funcionalidade, experiência e usabilidade. A funcionalidade leva em consideração as características técnicas do dispositivo, experiência foca no relacionamento entre o usuário e a tecnologia e usabilidade foca nas características de interação entre o usuário e o dispositivo. 

Os dispositivos serão avaliados quanto às especificações de funcionalidade, experiência e usabilidade. As análises serão feitas pela autora onde cada um dos controles será utilizado para interagir com a aplicação desenvolvida e as características pertinentes de cada um serão registradas e posteriormente comparadas. Os testes serão repetidos cinco vezes e o resultado levará em consideração todas as fases de testes. Parte das análises de usabilidade e experiência serão acessadas com base em questionários respondidos por X voluntários com idades variando entre X e X anos.

\section{Funcionalidade}
\label{funcionalidade}

Segundo \citeonline{mcnamara}, para avaliar a funcionalidade de um dispositivo pode-se analisar a performance, confiabilidade e durabilidade do mesmo. A quantidade de funções que um dispositivo oferece também deve ser considerada pois muitas opções de entrada podem inutilizar muitos comandos e poucas podem ser insuficientes. 

Quanto à performance, a característica a ser analisada será se o dispositivo oferece resposta rápida aos comandos do usuário, ou seja, se são observados atrasos na comunicação entre o dispositivo e o celular. A facilidade de conexão do controle ao dispositivo móvel e a preservação desta conexão serão características de confiabilidade. Por fim, a durabilidade levará em consideração o tipo de material de cada controle e se houve ou não falhas mecânicas na execução de comandos.

\section{Experiência}
\label{experiencia}

“(Experiência) remete à todas as qualidades do sistema interativo que o fazem memorável, satisfatório e gratificante.” \cite[tradução nossa]{benyon} Tendo em mente os controles de interação, usuários podem obter uma experiência negativa se as características de funcionalidades forem insatisfatórias e se houver problemas para encontrar os botões corretos no controle, já que neste caso poderá ser necessária a remoção do capacete de visualização resultando em uma interrupção da experiência em RV. 

\section{Usabilidade}

A usabilidade, segundo a ISO 9241-11 \cite{iso9241} tem como objetivo definir usabilidade e explica como identificar a informação necessária para avaliação de usabilidade de um computador em termos de medidas de desempenho e satisfação do usuário, ou seja, mede o quanto um usuário específico pode utilizar um produto e atingir os seus objetivos com eficácia, eficiência e satisfação em um contexto específico de uso.

De acordo com \citeonline{brown}, eficácia descreve a habilidade do usuário em realizar uma tarefa com a tecnologia. Eficiência considera os recursos utilizados para realizar a tarefa, podem ser esforço mental, esforço físico ou tempo. A satisfação mede o quanto a interação impactou o usuário, devendo ser extraída somente através de respostas do usuário. Os questionários aplicados tiveram como base a ISSO 9241-9 e o trabalho de \cite{lewis} que é citado na ISO 9241-11 e oferece modelos de questionário para avaliação da satisfação de um produto. Por fim, o contexto de uso leva em consideração não somente o ambiente físico mas também as características individuais dos usuários. 



