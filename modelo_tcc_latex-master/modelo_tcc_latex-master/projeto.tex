\documentclass[
	% -- opções da classe memoir --
	12pt,				% tamanho da fonte
	openright,			% capítulos começam em pág ímpar (insere página vazia caso preciso)
	oneside,			% para impressão em verso e anverso. Oposto a oneside
	a4paper,			% tamanho do papel. 
	% -- opções da classe abntex2 --
	%chapter=TITLE,		% títulos de capítulos convertidos em letras maiúsculas
	%section=TITLE,		% títulos de seções convertidos em letras maiúsculas
	%subsection=TITLE,	% títulos de subseções convertidos em letras maiúsculas
	%subsubsection=TITLE,% títulos de subsubseções convertidos em letras maiúsculas
	% -- opções do pacote babel --
	english,			% idioma adicional para hifenização
	french,				% idioma adicional para hifenização
	spanish,			% idioma adicional para hifenização
	brazil				% o último idioma é o principal do documento
	]{abntex2}

% ---
% Pacotes básicos 
% ---
\usepackage{lmodern}			% Usa a fonte Latin Modern			
\usepackage[T1]{fontenc}		    % Selecao de codigos de fonte.
\usepackage[utf8]{inputenc}		% Codificacao do documento (conversão automática dos acentos)
\usepackage{lastpage}			% Usado pela Ficha catalográfica
\usepackage{indentfirst}		    % Indenta o primeiro parágrafo de cada seção.
\usepackage{color}				% Controle das cores
\usepackage{graphicx}			% Inclusão de gráficos
\usepackage{microtype} 			% Para melhorias de justificação
\usepackage{afterpage}
\usepackage{amsmath}            % Pacote para fórmulas matemáticas
\usepackage{amssymb,url}
\usepackage{xcolor,tikz,bm,colortbl}
\usepackage[br]{nicealgo}       % Pacote para criação de algoritmos
\usepackage{customizacoes}
\usepackage{float}
\usepackage{subcaption}
\usepackage{longtable}

% ---
		
% ---
% Pacotes adicionais, usados apenas no âmbito do Modelo Canônico do abnteX2
% ---
\usepackage{lipsum}				% Para geração de dummy text
% ---

% ---
% Pacotes de citações
% ---
%\usepackage[brazilian,hyperpageref]{backref}	 % Paginas com as citações na bibl
\usepackage[alf, abnt-etal-list=0, abnt-repeated-author-omit=yes]{abntex2cite}	% Citações padrão ABNT

% --- 
% CONFIGURAÇÕES DE PACOTES
% --- 

% ---
% Configurações do pacote backref
\renewcommand{\familydefault}{\sfdefault}
% Usado sem a opção hyperpageref de backref
%\renewcommand{\backrefpagesname}{Citado na(s) página(s):~}
% Texto padrão antes do número das páginas
%\renewcommand{\backref}{}
% Define os textos da citação
%\renewcommand*{\backrefalt}[4]{
%	\ifcase #1 
%		Nenhuma citação no texto.
%	\or
%		Citado na página #2.
%	\else
%		Citado #1 vezes nas páginas #2.
%	\fi}
% ---

% ---
% Informações de dados para CAPA e FOLHA DE ROSTO
% ---
\titulo{AVALIANDO CONTROLES DE INTERAÇÃO EM APLICAÇÃO DE RV PARA \textit{SMARTPHONES}}
\autor{Amanda Gonçalves Dias}
\local{Bauru}
\data{2016}
\orientador{Prof. Dr. Wilson Yonezawa}
\instituicao{%
  Universidade Estadual Paulista
  \par
  Faculdade de Ciências
  \par
  Ciência da Computação}
\tipotrabalho{Trabalho de Conclusão de Curso}
% O preambulo deve conter o tipo do trabalho, o objetivo, 
% o nome da instituição e a área de concentração 
\preambulo{Trabalho de Conclusão de Curso do Curso de Ciência da Computação da Universidade Estadual Paulista ``Júlio de Mesquita Filho'', Faculdade de Ciências, Campus Bauru.}
% ---


% ---
% Configurações de aparência do PDF final

% alterando o aspecto da cor azul
\definecolor{blue}{RGB}{41,5,195}

% informações do PDF
\makeatletter
\hypersetup{
     	%pagebackref=true,
		pdftitle={\@title}, 
		pdfauthor={\@author},
    	pdfsubject={\imprimirpreambulo},
	    pdfcreator={LaTeX with abnTeX2},
		pdfkeywords={abnt}{latex}{abntex}{abntex2}{trabalho acadêmico}, 
		colorlinks=true,       		% false: boxed links; true: colored links
    	linkcolor=blue,          	% color of internal links
    	citecolor=blue,        		% color of links to bibliography
    	filecolor=magenta,      		% color of file links
		urlcolor=blue,
		bookmarksdepth=4
}
\makeatother
% --- 

% --- 
% Espaçamentos entre linhas e parágrafos 
% --- 

% O tamanho do parágrafo é dado por:
\setlength{\parindent}{1.3cm}

% Controle do espaçamento entre um parágrafo e outro:
\setlength{\parskip}{0.2cm}  % tente também \onelineskip

% ---
% compila o indice
% ---
\makeindex
% ---

% ----
% Início do documento
% ----
\begin{document}

% Seleciona o idioma do documento (conforme pacotes do babel)
%\selectlanguage{english}
\selectlanguage{brazil}

% Retira espaço extra obsoleto entre as frases.
\frenchspacing 

% ----------------------------------------------------------
% ELEMENTOS PRÉ-TEXTUAIS
% ----------------------------------------------------------
% \pretextual

% ---
% Capa
% ---
\imprimircapa
% ---

% ---
% Folha de rosto
% (o * indica que haverá a ficha bibliográfica)
% ---
\imprimirfolhaderosto
% ---

% ---
% Inserir a ficha bibliografica
% ---

% Isto é um exemplo de Ficha Catalográfica, ou ``Dados internacionais de
% catalogação-na-publicação''. Você pode utilizar este modelo como referência. 
% Porém, provavelmente a biblioteca da sua universidade lhe fornecerá um PDF
% com a ficha catalográfica definitiva após a defesa do trabalho. Quando estiver
% com o documento, salve-o como PDF no diretório do seu projeto e substitua todo
% o conteúdo de implementação deste arquivo pelo comando abaixo:
%
% \begin{fichacatalografica}
%     \includepdf{fig_ficha_catalografica.pdf}
% \end{fichacatalografica}

\begin{fichacatalografica}
	\ttfamily
	\vspace*{\fill}					% Posição vertical
	\hspace{1.5cm}
	\begin{centering}	% Minipage Centralizado
	\fbox{\begin{minipage}[c][7.5cm]{12.5cm}		% Largura
	\footnotesize
	\vspace{0.3cm}
	\hspace{2.0cm} Dias, Amanda Gonçalves.
	%Sobrenome, Nome do autor
	
	\hspace{2.0cm} \parbox[t]{\textwidth}{\hspace{0.5cm} Avaliando controles de interação em aplicação de RV \\ para \textit{smartphones}  / \imprimirautor, \imprimirdata} 	\\	
	
	\hspace{2.0cm} \parbox[t]{\textwidth}{\hspace{0.5cm} \pageref{LastPage} f. : il.} \\
	
	\hspace{2.0cm} \parbox[t]{\textwidth}{\hspace{0.5cm} \imprimirorientadorRotulo~\imprimirorientador} \\
	
	\hspace{2.0cm} \parbox[t]{\textwidth}{\hspace{0.5cm} Monografia (Graduação)~--~Universidade Estadual \\ Paulista. Faculdade de Ciências, Bauru, \imprimirdata \\}  
	\\	
	
	
	\hspace{2.0cm} \parbox[t]{\textwidth}{\hspace{0.5cm} 1. Realidade virtual. 2. Interação humano-computador. \\ 3. Análise de controles físicos. I. Universidade \\ Estadual Paulista. Faculdade de Ciências. II. Título.}	
	\end{minipage}}
	\end{centering}
\end{fichacatalografica}
% ---

% ---
% Inserir folha de aprovação
% ---

% Isto é um exemplo de Folha de aprovação, elemento obrigatório da NBR
% 14724/2011 (seção 4.2.1.3). Você pode utilizar este modelo até a aprovação
% do trabalho. Após isso, substitua todo o conteúdo deste arquivo por uma
% imagem da página assinada pela banca com o comando abaixo:
%
% \includepdf{folhadeaprovacao_final.pdf}
%
\begin{folhadeaprovacao}

  \begin{center}
    {\ABNTEXchapterfont\large\imprimirautor}

    \vspace*{\fill}\vspace*{\fill}
    \begin{center}
      \ABNTEXchapterfont\bfseries\Large\imprimirtitulo
    \end{center}
    \vspace*{\fill}
    
    \hspace{.45\textwidth}
    \begin{minipage}{.5\textwidth}
        \imprimirpreambulo
    \end{minipage}%
    \vspace*{\fill}
   \end{center}
        
   \center Banca Examinadora

   \assinatura{\textbf{\imprimirorientador} \\ Departamento de computação - Faculdade de Ciências - UNESP - Bauru} 
   \assinatura{\textbf{Profa. Dra. Simone das Graças Domingues Prado} \\ Departamento de computação - Faculdade de Ciências - UNESP - Bauru}
   \assinatura{\textbf{Prof. Dr. Humberto Ferasoli Filho} \\ Departamento de computação - Faculdade de Ciências - UNESP - Bauru}
   %\assinatura{\textbf{Professor} \\ Convidado 3}
   %\assinatura{\textbf{Professor} \\ Convidado 4}
      
   \begin{center}
    \vspace*{0.5cm}
    \par
    {Bauru, \_\_\_\_\_ de \_\_\_\_\_\_\_\_\_\_\_ de \_\_\_\_.}
    \vspace*{1cm}
  \end{center}
  
\end{folhadeaprovacao}
% ---

% ---
% Dedicatória
% ---
%\begin{dedicatoria}
%   \vspace*{\fill}
%   \centering
%   \noindent
%   \textit{Espaço destinado à dedicátoria do texto.} \vspace*{\fill}
%\end{dedicatoria}
% ---

% ---
% Agradecimentos
% ---
\begin{agradecimentos}
Agradeço à minha família por todo apoio e por terem me motivado a sempre buscar o melhor, servindo como a minha base e guia durante toda a minha vida. Agradeço ao meu namorado por toda paciência e ajuda que me deu durante toda a minha graduação. Agradeço aos meus amigos que me motivaram a continuar e que contribuíram de forma direta para a conclusão desta monografia. Por fim, agradeço também ao meu orientador, que me ensinou a escrever um texto acadêmico e me mostrou que é sempre possível fazer melhor.
\end{agradecimentos}
% ---

% ---
% Epígrafe
% ---
\begin{epigrafe}
    \vspace*{\fill}
	\begin{flushright}
		\textit{"We keep moving forward, opening new doors, and doing new things, because we're curious and curiosity keeps leading us down new paths." (Walt Disney)}
	\end{flushright}
\end{epigrafe}
% ---

% ---
% RESUMOS
% ---

% resumo em português
\setlength{\absparsep}{18pt} % ajusta o espaçamento dos parágrafos do resumo
\begin{resumo}

Com a introdução da Realidade Virtual (RV), questões sobre a interação com o ambiente virtual entraram em evidência. Dispositivos utilizados diariamente com eficiência e eficácia em aplicações \textit{desktop} comuns, não apresentam os mesmos benefícios para aplicações em RV. 
Este trabalho teve como objetivo avaliar controles de interação em uma aplicação de RV desenvolvida para \textit{smartphones} onde a escolha dos controles foi feita principalmente com base em seu meio de conexão, sendo estudados controles que se conectam ao \textit{smartphone} através da conexão Bluetooth, via cabo e o próprio toque na tela. 
As avaliações realizadas sob os controles avaliaram a capacidade de interação destes com base em sua funcionalidade, usabilidade e experiência. Ademais, foram aplicados os questionários \textit{"The After-Scenario Questionnaire (ASQ)"} e o \textit{"The Post-Study System Usability Questionnaire (PSSUQ)"} a fim de acessar as características de usabilidade e experiência dos usuários. 

\textbf{Palavras-chave:} Realidade Virtual, Interação Humano-Computador, Análise de controles físicos.
\end{resumo}

% resumo em inglês
\begin{resumo}[Abstract]
 \begin{otherlanguage*}{english}
 
The Virtual Reality (VR) introduction brought into evidence concerns about the interaction with the virtual environment. Devices that are daily used with efficiency and efficacy within usual desktop applications, do not bring the same benefits within VR aplications.
This paper aimed to evaluate external controllers using a VR application developed for smartphones where the choice of the controllers were made mainly by their communication mode, that were Bluetooth, cable and touch. 
The evaluation performed on the controllers measured their capacity of interaction according to their functionality, usability and experience. Futhermore, users responded to two questionnaires: The After-Scenario Questionnaire (ASQ) and The Post-Study System Usability Questionnaire (PSSUQ) in order to access the user's usability and experience with the controller.

\textbf{Keywords:} Virtual Reality, Human-Computer Interaction, Physical controllers evaluation.
 
 \end{otherlanguage*}
\end{resumo}
% ---

% ---
% inserir lista de ilustrações
% ---
\pdfbookmark[0]{\listfigurename}{lof}
\listoffigures*
\cleardoublepage
% ---

% ---
% inserir lista de tabelas
% ---
\pdfbookmark[0]{\listtablename}{lot}
\listoftables*
\cleardoublepage
% ---

% ---
% inserir lista de abreviaturas e siglas
% ---
% ---

% ---
% inserir o sumario
% ---
\pdfbookmark[0]{\contentsname}{toc}
\tableofcontents*
\cleardoublepage
% ---



% ----------------------------------------------------------
% ELEMENTOS TEXTUAIS
% ----------------------------------------------------------
\pagestyle{simple}

% ----------------------------------------------------------
% Introdução (exemplo de capítulo sem numeração, mas presente no Sumário)
% ----------------------------------------------------------

\chapter{Introdução}
\label{c.introducao}

“Com o advento da realidade virtual e o avanço dos recursos computacionais, as representações interativas e imersivas do imaginário, bem como a reprodução do real, tornaram-se mais fáceis de serem obtidas.” \cite[p. ~9]{torilivro}

A realidade virtual (RV) vem ganhando espaço em diversos setores como jogos, indústria e educação. Na área de jogos, empresas como Playstation® e Oculus® oferecem um acervo de jogos para as suas respectivas plataformas. Ao procurar por jogos em RV na Google Play \cite{googleplay}, encontram-se algumas opções fornecidas por diversas empresas.

Na indústria, a RV pode ser utilizada para avaliar o design de um produto antes do mesmo ser produzido. A montadora \textit{The Ford Motor Company} é uma das empresas que utilizam a realidade virtual. Com esta tecnologia, é possível visualizar virtualmente tanto o exterior como o interior de um carro a ser produzido e avaliar aspectos de engenharia e design. “A realidade virtual pode ser mais efetiva do que desenvolver o design no mundo real. Só neste ano, designers e engenheiros verificaram mais de 135.000 detalhes em 193 protótipos virtuais de veículos” \cite[tradução nossa]{ford}. 

Já na área da educação, a realidade virtual pode ser aplicada por meio de jogos educativos e aulas imersivas. Imagine utilizar a RV em uma aula de história para simular um acontecimento histórico, ou ter a sensação de estar no espaço enquanto assiste uma aula de astronomia. Pesquisas como \cite{youngblut} e \cite{carvalho} mostram como a realidade virtual pode ser incorporada na escola.

Para se obter uma experiência em realidade virtual, são necessários capacetes de visualização ou óculos de RV, um \textit{display} por onde a aplicação irá rodar, um dispositivo de interação e a aplicação em RV. 

Atualmente existem vários modelos de óculos de RV com suporte à realidade virtual, tais como: Oculus Rift da Oculus® com preço estimado em R\$ 4.620,90 e Samsung Gear VR da Samsung® (R\$ 799,00). No entanto, o Google Cardboard da Google® é o que possui preço mais acessível em torno de R\$ 21,97 possuindo atualmente duas versões, cada uma com um meio de interação próprio. Em novembro 2016, a Google® lançou um novo visualizador denominado Daydream (Figura ~\ref{f.daydream}). Este visualizador acompanha um controle com comunicação \textit{Bluetooth} e o capacete de visualização feito com tecido para garantir maior conforto, além de um suporte para fixação do visualizador à cabeça do usuário. Os \textit{smartphones} compatíveis com o Daydream são os que possuem Android 7.0 (Nougat) ou superior. Outras versões do Android, apesar de serem compatíveis com aplicações em RV, não oferecem suporte para o controle do Daydream.

\begin{figure}[ht]
	\caption{\small Daydream}
	\centering
	\includegraphics[scale=0.15]{Imagens/daydream.jpg}
	\label{f.daydream}
	\legend{\small Fonte: \cite{daydream}}
\end{figure}

As aplicações em RV podem ser desenvolvidas tanto para computadores como para dispositivos móveis. A principal diferença entre as duas plataformas é a capacidade de processamento e memória, que são inferiores nos dispositivos móveis. Para as aplicações \textit{desktop} utiliza-se óculos de RV como o Oculus Rift para executar a aplicação. Já nas aplicações para dispositivos móveis, é necessário um visualizador como o Google Cardboard por onde o \textit{smartphone} será encaixado.

Em todos os casos, uma aplicação em realidade virtual deve permitir algum tipo de interação. Para isso, com o visualizador posicionado na cabeça do usuário, é possível utilizar a movimentação da cabeça e um controle externo como luvas, mouse 3D, \textit{joystick}, entre outros. A grande dificuldade é encontrar uma forma de interação que seja viável e que garanta usabilidade e experiência satisfatórias aos usuários. “A necessidade de se fazer uso de aparatos tecnológicos para a interação do usuário com o ambiente virtual provoca restrições, tanto pelo aspecto econômico e tecnológico, quanto pelo desconforto, mas permite ao usuário fazer coisas que antes eram impossíveis ou inviáveis. ” \cite[p. ~3]{torilivro}

Este projeto utiliza óculos de RV e um dispositivo móvel para criar uma aplicação que explora os conceitos da realidade virtual visando explorar diferentes formas de interação utilizando controles externos.

\section{Problema}
\label{s.problema}

Apesar da quantidade de aplicações desenvolvidas utilizando RV para dispositivos móveis estar em crescimento, ela é pequena em comparação com aplicações para \textit{smartphones} sem a tecnologia de RV. Além disso, os estudos de formas de interação com aplicações em RV que possam propiciar conforto, eficácia e conectividade adequada com \textit{smartphones} são atuais. 

O Google Cardboard versão 1 propõe apenas duas formas de interação com o usuário: Movimentação da cabeça e um par de ímãs que, quando utilizados representam um toque na tela, seu funcionamento é ilustrado na Figura ~\ref{f.cardboard1}. Já a versão 2 não possui ímãs e utiliza o toque na tela como forma de interação como é ilustrado nas Figuras ~\ref{f.botaocardboard1} e ~\ref{f.botaocardboard2}. Atualmente, o suporte para a interação via ímã está sendo descontinuado e, por isso, as análises foram feitas utilizando o Google Cardboard versão 2.

\begin{figure}[H]
	\caption{\small Funcionamento do ímã}
	\centering
	\includegraphics[scale=0.7]{Imagens/cardboard.png}
	\label{f.cardboard1}
	\legend{\small Fonte: Elaborada pelo autor.}
\end{figure}

\begin{figure}[H]
	
	\begin{minipage}{.5\textwidth}{
			\centering
			\captionof{figure}{\small Botão pressionado}
			\includegraphics[height=5cm]{Imagens/botaocardboard1.jpg}		
			\label{f.botaocardboard1}
			\legend{\small Fonte: Elaborada pelo autor.}	
		}
	\end{minipage}
	\begin{minipage}{.5\textwidth}{
			\centering
			\captionof{figure}{\small Liberação do botão (toque na tela)}
			\includegraphics[height=5cm]{Imagens/botaocardboard2.jpg}		
			\label{f.botaocardboard2}	
			\legend{\small Fonte: Elaborada pelo autor.}
		}
	\end{minipage}
\end{figure}

Contar apenas com duas formas de entrada de dados, limita as opções de interação do usuário com a aplicação. A empresa Oculus®, vende seus óculos de RV juntamente com um controle de vídeo game, o que garante maior flexibilidade na criação de aplicações em RV. No entanto, a diferença de preço entre o Google Cardboard e o Oculus Rift é significativa, isto é,  o primeiro custa aproximadamente duzentas vezes o preço do segundo.

Encontrar formas de interação ótimas para um determinado ambiente é um problema muito estudado na área de Interação Humano-Computador pois, com os avanços tecnológicos, surgem novas possibilidades de interação. É o caso da realidade virtual, onde o usuário utiliza o próprio corpo como forma de entrada de dados. Também é possível utilizar um controle externo para ações adicionais, no entanto, a escolha deste controle precisa ser estudada.

\begin{citacao}
A grande maioria dos usuários não consegue dominar todas as funcionalidades de um sistema antes que mais funções sejam adicionadas, isso acontece porque com a evolução tecnológica caminhando a passos largos, as novas possibilidades saltam aos olhos dos fabricantes de hardware e desenvolvedores de software, celulares equipados com GPS, câmeras digitais com reconhecimento facial, telas sensíveis ao toque, jogadores controlando jogos apenas movimentando o próprio corpo, interação com objetos virtuais através da realidade aumentada, interação ativa do telespectador com a televisão. Todas essas tecnologias desafiam os fabricantes a criar interfaces que possibilitem o acesso a essas novidades sem perder a objetividade e a clareza. \cite{oliveira}
\end{citacao}

\section{Justificativa}
\label{s.justificativa}

A realidade virtual é um tema em expansão, objeto de pesquisa de grandes empresas como o Facebook® e a Google®. Logo, investigar formas de utilização desta tecnologia em dispositivos móveis como \textit{smartphones} é contribuir com conhecimento para a área bem como auxiliar o trabalho de desenvolvedores. 

É interessante apresentar formas diferentes de interação com aplicações em RV e, ao mesmo tempo, obter uma experiência em realidade virtual acessível. O sistema operacional Android oferece bibliotecas que ajudam no tratamento de diversos meios de entradas como \textit{Bluetooth} e via cabo. Além disso, os \textit{smartphones} atuais possuem sensores como acelerômetro e giroscópio que são utilizados para o desenvolvimento de aplicações em RV. Com o auxílio dos recursos mencionados, é possível realizar uma comparação entre diversos tipos de controles físicos, a fim de escolher o mais apropriado para ser utilizado em uma aplicação em RV. 

Por fim, o projeto pode ser realizado com materiais de baixo custo e de fácil acesso, podendo ser encontrados softwares completamente gratuitos.

\section{Objetivos}
\label{s.objetivos}

\subsection{Objetivo Geral}
\label{s.geral}
Explorar formas de interação do usuário em aplicação de RV executada em dispositivos móveis por intermédio de um controle externo.

\subsection{Objetivos Específicos}
\label{s.especifico}
- Identificar os elementos necessários para a criação de uma aplicação em RV.

- Definir as principais restrições na escolha e ferramentas de RV para construção da aplicação.

- Identificar dispositivos físicos como forma de controle na interação entre o usuário e a aplicação RV.

- Realizar a análise dos controles físicos.

\chapter{Fundamentação Teórica}
\label{c.fundamentacao}

\section{Realidade Virtual}
\label{s.rv}
A realidade virtual “é uma interface avançada para aplicações computacionais, que permite ao usuário a movimentação (navegação) e interação em tempo real, em um ambiente tridimensional, podendo fazer uso de dispositivos multissensoriais, para atuação ou feedback.” \cite[p. ~7]{torilivro}

A tecnologia de RV permite a imersão do usuário em um ambiente virtual através de um sistema computacional. Com esta tecnologia, é possível enganar os sentidos do usuário para que o mesmo tenha a sensação de que está dentro do ambiente simulado. 

É importante diferenciar os conceitos de RV, realidade aumentada (RA) e virtualidade aumentada (VA). Como explica \cite{milgram}, o ambiente propiciado pela RV é totalmente sintético, ou virtual, e pode ou não imitar o mundo real. Já ambientes em RA intercalam o real e o virtual tornando o capacete de visualização “transparente”, acrescentando elementos virtuais no ambiente real. Por fim, a VA traz ambientes virtuais com alguns elementos reais como a representação das mãos do usuário, por exemplo. A figura X mostra uma relação entre as tecnologias descritas acima.

\begin{figure}[ht]
	\caption{\small Virtual Reality Continuum}
	\centering
	\includegraphics[scale=0.7]{Imagens/milgram.png}
	\label{f.milgram}
	\legend{\small Fonte: \cite{milgram}}
\end{figure}

Para se criar a sensação de imersão propiciada pela RV, é utilizada a estereoscopia (Figura X), ou seja, é gerada uma imagem para cada olho. Esta técnica faz com que o cérebro interprete as imagens como uma. \citeonline[p. ~221]{siscoutto} explicam que a importância da estereoscopia ou visão binocular pode ser averiguada na prática ao fechar um olho e tentar exercer atividades cotidianas desta forma. O que será observado é que a visão monocular, torna o simples ato de pegar um objeto sobre a mesa uma tarefa difícil pois esta visão conta com uma percepção rudimentar de profundidade.

\begin{figure}[H]
	\caption{\small Estereoscopia}
	\centering
	\includegraphics[scale=0.7]{Imagens/estereoscopia.png}
	\label{f.estereoscopia}
	\legend{\small Fonte: Elaborada pelo autor.}
\end{figure}

Além da estereoscopia, a navegação e a interação com o ambiente virtual também são características da RV. De acordo com \citeonline[p. ~9]{torilivro}, a navegação em espaços tridimensionais dá-se por movimentos de translação e de rotação ao longo dos três eixos (X, Y, Z) resultando em 6 graus de liberdade (3 de rotação e 3 de translação) como mostra a figura X. 

\begin{figure}[ht]
	\caption{\small Graus de Liberdade}
	\centering
	\includegraphics[scale=1.0]{Imagens/grausliberdade.png}
	\label{f.grausliberdade}
	\legend{\small Fonte: \cite{torilivro}.}
\end{figure}

O conceito de realidade virtual, segundo \citeonline[p. ~98]{rodrigues}, apesar de existir a mais de vinte anos, tem ganhado popularidade apenas recentemente. Este fato se deu devido a diminuição do custo para a sua implementação.

\subsection{Fatores Fisiológicos}
\label{ss.fatoresfisiologicos}
Quando tenta-se criar uma simulação imersiva, é importante ter em mente alguns fatores fisiológicos que não são levados em consideração em aplicações não imersivas. Estes fatores, quando não implementados corretamente, podem levar ao chamado motion sickness ou VR sickness que, segundo a \citeonline{vrsickness}, é a sensação de náusea devido à disparidade entre o que é sentido e o que é esperado sentir. Esta disparidade faz com que nosso corpo se sinta “envenenado” causando desconforto ao usuário. 

De acordo com a \citeonline{oculus}, encontrar fatores que contribuem para o motion sickness pode ser complicado pois as pessoas sentem este desconforto de forma desigual. Enquanto um pode sentir náusea em um curto período de exposição à uma aplicação em RV, outro pode não sentir nada após um longo período. Outro empecilho é a possibilidade de se acostumar ao ambiente imersivo e não sentir mais o desconforto. Por isso, não é recomendado que os desenvolvedores da aplicação analisem os fatores causadores do VR sickness.

Apesar da dificuldade, algumas características foram consideradas fatores contribuintes para o desconforto do usuário. Um destes fatores é a aceleração da câmera. Diferente de aplicações não imersivas, a movimentação do usuário pela aplicação deve ser analisada com cuidado pois, como o usuário está parado enquanto o personagem está se movimentando, ocorre a discrepância de sensações resultando no desconforto.

Outro fator é o grau de controle do usuário sobre o ambiente. Ao travar a tela para que alguma mensagem seja mostrada ou realizar a rotação da câmera para algum ponto, é retirado o poder de controle do usuário, o que também é considerado um fator facilitador do VR sickness.

A distância dos objetos à câmera também deve ser analisada. O usuário deve ser capaz de ler uma mensagem em frente à câmera com facilidade, o posicionamento incorreto da mensagem pode causar incômodo ao usuário dificultando a leitura.

Tendo em vista a redução do VR sickness foram criados guias de boas práticas para serem levados em consideração ao desenvolver aplicações em RV.

\subsection{Boas Práticas}
\label{ss.boaspraticas}

Para auxiliar desenvolvedores, a Google® lançou um aplicativo para dispositivos Android denominado Cardboard Design Lab (Figura X). Neste aplicativo, o usuário pode aprender técnicas que são classificadas em duas categorias: criação e imersão. A primeira categoria, segundo \citeonline{hopkins}, foca nos princípios básicos da criação de aplicações em RV. Já a segunda categoria é mais exploratória, levando em consideração teoria e experiências através do aplicativo. Em cada categoria são fornecidas lições onde o usuário verá cenários onde as técnicas são aplicadas e, em alguns casos, sentirá o motion sickness quando determinado cuidado não for tomado. 

\begin{figure}[h]
	\caption{\small Cardboard Design Lab}
	\centering
	\includegraphics[scale=0.7]{Imagens/designlab.png}
	\label{f.designlab}
	\legend{\small Fonte: \cite{designlab}}
\end{figure}

Primeiramente, o aplicativo mostra a necessidade de se utilizar o retículo - círculo no centro da tela que aumenta o seu raio ao passar por algum objeto, variando o seu posicionamento com base na movimentação de cabeça do usuário. Como ainda não existe o rastreamento do olho do usuário em aplicações de RV para dispositivos móveis, é muito difícil saber para onde a cabeça está apontando quando o retículo é inexistente, tornando a seleção de objetos em um cenário muito mais complicada. 

A distância entre o usuário e caixas de textos que eventualmente aparecem na tela é outro fator a ser levado em consideração. Segundo o mesmo aplicativo, uma distância confortável para a leitura é de três metros à frente da câmera. 

A movimentação do usuário no cenário pode ser feita de algumas maneiras para que não se cause o desconforto. Um modo é mover a câmera de forma constante, sem variações de aceleração. Também é possível realizar o desvanecer ou fade da tela, que é quando a imagem do cenário desaparece e outro cenário reaparece. Segundo \citeonline{tom}, se este clique da tela estiver na frequência correta, o nosso cérebro interpreta esta mudança como um piscar dos olhos, tornando a transição imperceptível. 

Como um dos conceitos da realidade virtual é a imersão do usuário, o mesmo deve se sentir confortável no ambiente e, para que isso ocorra, a escala dos objetos na aplicação deve corresponder ao máximo o real, para que o usuário se sinta no mesmo “mundo” da aplicação. O mesmo vale para os sons presentes na aplicação, que devem considerar o posicionamento do usuário. Além disso, utilizar pontos de referência no ambiente evita que o usuário se sinta desorientado. Quanto ao controle do usuário sobre a aplicação, uma recomendação geral é nunca travar a cena. Ou seja, é importante que a movimentação da cabeça esteja sempre habilitada.

Para auxiliar o usuário a navegar através da aplicação, pode-se utilizar recursos como pontos de luz para indicar o caminho a ser seguido, além de sinalizações para indicar onde o usuário deveria olhar. A última recomendação do aplicativo apresentado pela Google é criar aplicações visualmente agradáveis como na Figura X a fim de maximizar a ilusão de imersão do usuário. 

\begin{figure}[h]
	\caption{\small Visual Agradável}
	\centering
	\includegraphics[scale=0.2]{Imagens/makeitbeautiful.png}
	\label{f.makeitbeatiful}
	\legend{\small Fonte: \cite{hopkins}.}
\end{figure}


\section{Dispositivos de Controle de Interação Homem/Máquina}
\label{s.dispositivos}

Com a inserção das máquinas no cotidiano das pessoas, surgiu uma área de estudos multidisciplinar que visa compreender as interações entre o usuário e o computador denominada Human Computer Interaction. Segundo \citeonline[p. ~3]{dix}, entende-se por usuário um indivíduo ou grupo de pessoas trabalhando juntas ou uma sequência de usuários em uma organização. O usuário é qualquer um que está tentando cumprir um objetivo utilizando a tecnologia. O computador representa qualquer tecnologia inclusive partes não computadorizadas como outras pessoas. Por interação entende-se qualquer meio de comunicação entre o usuário e a máquina.  

Em relação a área de estudo interação homem-computador, é possível citar um grande número de dispositivos que podem ser considerados computador pelos quais o usuário irá interagir a fim de realizar algum objetivo. Como exemplos, pode-se citar os dispositivos de entrada como dispositivos de texto, de posicionamento e indicação e os dispositivos de saída como a tela do computador ou celular. 

Como exemplos de dispositivos de entrada de texto, o teclado é um dos mais comuns. De acordo com \citeonline{dix}, os teclados funcionam com o pressionar das teclas, fechando a conexão e fazendo com que um código de caractere seja enviado ao computador. Atualmente, a maioria dos teclados seguem o layout de teclas QWERT que, apesar de não ser o layout ótimo, é utilizado pois o teclado foi baseado nas máquinas de escrever onde o posicionamento das letras desta forma (letras que possivelmente seriam combinadas eram colocadas distantes no teclado), evitava que os braços se aglomerassem de um lado da máquina. Reconhecedor de voz e de escrita também podem ser considerados dispositivos de entrada de texto.

\citeonline{dix} descrevem os dispositivos de posicionamento e indicação dando o exemplo do mouse como um destes dispositivos. Outros dispositivos como os joysticks, telas touchscreens, touchpads e as setas do teclado também estão nesta categoria.  Quando o contexto é um ambiente 3D, o mouse 3D, as luvas de dados (luvas com fibras ópticas ao longo dos dedos que detectam o ângulo das articulações) e capacetes de realidade virtual são alguns dos dispositivos desta categoria. 

Assim como os computadores desktops, outros dispositivos de interação homem-máquina também tiveram a sua evolução ao longo do tempo. As telas touchscreens sendo utilizadas como mouse em celulares e desktops, o desenvolvimento de telas 3D e capacetes em realidade virtual capazes de criar ambientes imersivos são exemplos desta evolução. O crescimento dos dispositivos móveis acarretou no aumento do número de dispositivos de interação com estes minicomputadores como telas de alta densidade, capacidade gráfica 3D e sensores. Com a anexação de sensores como o acelerômetro e giroscópio foi possível a incorporação de aplicações em realidade virtual em dispositivos móveis.

\subsection{Acelerômetro e Giroscópio}
\label{ss.acelerometrogiroscopio}

Segundo \citeonline{bergstrom}, os sensores de inércia como os acelerômetros e os giroscópios possuem a função de converter um fenômeno físico em um sinal mensurável. O acelerômetro é normalmente definido num plano cartesiano e mede a força cinética causada por uma aceleração linear. Já os giroscópios medem a velocidade angular de uma rotação sob seu eixo primário.

De acordo com a empresa \citeonline{dimension}, a aceleração medida pelo acelerômetro pode ser estática ou dinâmica. A gravidade é um exemplo de força estática e uma das vantagens de medir esta aceleração é a possibilidade de descobrir o ângulo do dispositivo em relação à terra. Já a medição da aceleração dinâmica revela em qual direção o dispositivo está se movendo. 

Quando inseridos em smartphones, os acelerômetros podem ser utilizados para funções diversas que vão desde realizar a rotação da tela de acordo com a orientação em que o dispositivo se encontra até reconhecer movimentos do usuário como o caminhar e a movimentação da cabeça. 

Já os giroscópios funcionam como uma bússola indicando a posição do dispositivo no espaço. Segundo a empresa \citeonline{android}, os giroscópios medem a taxa de rotação ao longo dos três eixos do sensor, onde a rotação é positiva no sentido anti-horário. Aplicações em RV utilizam as informações do giroscópio para saber para onde o usuário está olhando através da rotação da cabeça.

\subsection{Computação Móvel}
\label{ss.computacaomovel}

Tendo em vista a popularidade e o avanço dos dispositivos móveis, é importante entender o significado de computação móvel a fim de se criar aplicações que podem ser executadas neste contexto. Para entender o conceito de computação móvel, é preciso entender o que é mobilidade.

“No contexto da computação móvel, mobilidade se refere ao uso pelas pessoas de dispositivos móveis portáveis funcionalmente poderosos que ofereçam a capacidade de realizar facilmente um conjunto de funções de aplicação, sendo também capazes de conectar-se, obter dados e fornecê-los a outros usuários, aplicações e sistemas. ”. \cite[p. ~1]{lee}

Com os avanços do hardware dos computadores, foi possível criar aparelhos cada vez menores e, apesar dos dispositivos móveis serem portáteis, diferentes aparelhos possuem diferentes níveis de portabilidade. De acordo com \citeonline{lee}, a portabilidade é afetada pelos fatores tamanho e peso do dispositivo (considerando seus acessórios). Logo, um smartphone que cabe em uma mão é mais portátil do que um laptop, por exemplo. 

Contudo, além da portabilidade também é necessário levar em consideração a usabilidade, a funcionalidade e a conectividade dos dispositivos. \citeonline{lee} definem a usabilidade como dependente do usuário, ambiente e as características do dispositivo enquanto que a funcionalidade é dividida nas categorias aplicações independentes, ou seja, o usuário não tem contato com outro usuário ou sistema, e dependentes (é necessário conectar-se a outro usuário ou sistema). Quanto à conectividade é importante apontar que um dispositivo móvel não possui necessariamente uma conexão sem fio.

Quanto à usabilidade é natural que um dispositivo móvel como um laptop seja mais facilmente transportado por um adulto do que por uma criança. Assim como certos usuários não possuem facilidade para interagir com certos dispositivos. Outros fatores como a característica do ambiente e as características do dispositivo afetam na escolha do dispositivo com melhor usabilidade. 

Em resumo, a computação móvel é definida a partir de diversos fatores. Diferentes tipos de dispositivos móveis possuem características distintas que, apesar de serem consideradas móveis, apresentam certas diferenças que devem ser levadas em consideração para se escolher o dispositivo mais adequado para determinada aplicação.

Smartphones são dispositivos móveis que estão se tornando cada vez menores e mais potentes. A evolução dos smartphones permitiu a conexão de diversos dispositivos que aumentam ainda mais as possibilidades de interação homem-máquina. A seguir, são apresentados alguns dispositivos que serão estudados neste trabalho e que podem ser utilizados como controles externos e engrandecer as possibilidades de ações em aplicações para smartphones. 

\subsection{Controles}

\subsubsection{Controle Google Cardboard 2.0}
\label{ss.cardboard}

Como mencionado anteriormente, o Google Cardboard 2.0 utiliza o toque na tela como forma de interação. Apesar de ser totalmente feito em papelão, este visualizador é montado de forma que ao pressionar um botão na parte superior do mesmo, o toque na tela é realizado. Como smartphones não reconhecem o contato do papelão na tela como um toque, a parte que realiza o contato é revestido por tecido condutivo.

\subsubsection{Controle via Cabo}
\label{ss.cabo}

Para a conexão via cabo, é necessário somente um cabo OTG (Figura 5) e um controle de vídeo game que possua interface USB. De acordo com a empresa \citeonline{otg}, USB On-The-Go (OTG) foi criado em 2001 e seu objetivo é aumentar a capacidade de dispositivos móveis adicionando funcionalidade de host para conexões com periféricos com conexão USB. Esta tecnologia foi necessária para atualizar o padrão USB focando dispositivos móveis. A ideia é que ao conectar um periférico ao cabo OTG, o mesmo funcione automaticamente sem a necessidade de configurações adicionais.
Exemplos de dispositivos compatíveis são controles da Playstation® e Xbox®. No caso dos controles mais antigos da Playstation® que não possuem cabos USB, é possível adquirir um adaptador como o da Figura 6.

\begin{figure}[H]
	
	\begin{minipage}{.5\textwidth}{
		\centering
		\captionof{figure}{\small Cabo USB OTG}
		\includegraphics[height=4cm]{Imagens/otg.png}		
		\label{f.otg}	
		\legend{\small Fonte: Elaborada pelo autor.}
	}
	\end{minipage}
	\begin{minipage}{.5\textwidth}{
			\centering
			\captionof{figure}{\small Adaptador USB}
			\includegraphics[height=4cm]{Imagens/adaptador.png}		
			\label{f.adaptador}
			\legend{\small Fonte: Elaborada pelo autor.}	
		}
	\end{minipage}
	
\end{figure}

\subsubsection{Controle Bluetooth}

A tecnologia Bluetooth está presente em diversos dispositivos ao redor do mundo, inclusive nos telefones celulares. 

\begin{quote}
Um dispositivo Bluetooth utiliza ondas de rádio ao invés de cabos para se conectar a um celular ou computador. Um produto Bluetooth como um fone de ouvido ou relógio, possui um pequeno chip de computador com rádio Bluetooth e software que facilita a conexão. Quando dois dispositivos Bluetooth querem trocar informações, eles precisam ser pareados. \cite[tradução nossa]{bluetooth}
\end{quote}

Para este projeto, serão explorados dois dispositivos Bluetooth que podem ser utilizados como dispositivos de controle do usuário com a aplicação em realidade virtual. O primeiro consiste em um teclado de computador que utiliza o Bluetooth como forma de comunicação. Atualmente é possível encontrar este tipo de teclado que, além de se comunicar com o computador, também se conecta com dispositivos móveis sem que seja necessário a instalação de softwares específicos. 

O outro dispositivo a ser utilizado será um joystick que utiliza a comunicação Bluetooth. Atualmente encontra-se no mercado alguns controles com esta característica como o Wii Remote, controle que acompanha o console Nintendo Wii® (Figura X), e o controle da VR Box (Figura X) que acompanha o capacete de visualização da mesma empresa.

\begin{figure}[H]
	
	\begin{minipage}{.5\textwidth}{
			\centering
			\captionof{figure}{\small Wii Remote}
			\includegraphics[width=.7\linewidth]{Imagens/wiiremote.png}		
			\label{f.wiiremote}	
			\legend{\small Fonte: Elaborada pelo autor.}
		}
	\end{minipage}
	\begin{minipage}{.5\textwidth}{
			\centering
			\captionof{figure}{Adaptador USB}
			\includegraphics[width=.7\linewidth]{Imagens/controlevrbox.jpg}		
			\label{f.controlevrbox}
			\legend{\small Fonte: Elaborada pelo autor.}	
		}
	\end{minipage}
	
\end{figure}

Para este projeto, o joystick utilizado será o fornecido pela empresa VR Box pois o mesmo permite a conexão com dispositivos móveis diferentemente do Wii Remote que não possui suporte para este tipo de conexão. 	
\chapter{Desenvolvimento do Trabalho}
\label{c.desenvolvimento}

\section{Modelo de Pesquisa}
\label{s.metodo}

A Figura ~\ref{f.metodopesquisa} apresenta um diagrama com os principais elementos deste projeto de pesquisa. As elipses representam temas ou assuntos. Os retângulos com cantos arredondados, atividades. As setas representam as relações entre diferentes elementos. Os cilindros representam bases de dados de artigos científicos.

\begin{figure}[H]
	\caption{\small Modelo de Pesquisa.}
	\centering
	\includegraphics[scale=0.7]{Imagens/metodologia.png}
	\label{f.metodopesquisa}
	\legend{\small Fonte: Elaborada pelo autor.}
\end{figure}

A pesquisa foi dividida em seis etapas. A princípio (fundamentação teórica) foi feito um levantamento bibliográfico sobre os tipos de controle físicos que podem ser utilizados no celular, ferramentas para o desenvolvimento de aplicações em realidade virtual e métodos de avaliação para controles físicos.

A segunda etapa (preparação do ambiente operacional), envolveu a escolha dos \textit{software} a serem utilizados com base na exequibilidade do projeto e da acessibilidade das ferramentas, ou seja, foram capazes de proporcionar as vias necessárias para o êxito do projeto de forma gratuita e com documentação clara. 

Na terceira fase do projeto (escolha dos controles físicos), foi feita a escolha de três tipos de controles que apresentam três diferentes tipos de conexão e interatividade: via cabo, \textit{Bluetooth} e toque na tela. 

Na quarta etapa (desenvolvimento da aplicação em RV), foi realizada a construção da aplicação, juntamente com os testes e correções considerando a usabilidade da mesma.

A quinta etapa (análise dos controles), verificou se o projeto atingiu os objetivos geral e específicos propostos levando em consideração a análise dos controles em relação as suas características de usabilidade, experiência e funcionalidade. 

Na última (sexta) etapa (relatório de pesquisa), foi elaborado o relatório final da pesquisa registrando todos os procedimentos realizados bem como os resultados e conclusões.

\section{Controles}

\subsection{Controle Google Cardboard 2.0}
\label{ss.cardboard}

Como mencionado anteriormente, o Google Cardboard 2.0 utiliza o toque na tela como forma de interação. Apesar de ser totalmente feito em papelão, este visualizador é montado de forma que ao pressionar um botão na parte superior do mesmo, o toque na tela é realizado. Como \textit{smartphones} não reconhecem o contato do papelão na tela, a parte que realiza o contato é revestido por tecido condutivo.

\subsection{Controle via Cabo}
\label{ss.cabo}

Para a conexão via cabo, é necessário somente um cabo \textit{On-The-Go} (OTG) (Figura ~\ref{f.otg}) e um controle de vídeo game que possua interface USB. O USB OTG foi criado em 2001 e seu objetivo é aumentar a capacidade de dispositivos móveis adicionando funcionalidade de \textit{host} para conexões de periféricos com conexão USB. A ideia é a de, ao conectar um periférico ao cabo OTG, o mesmo funcione automaticamente sem a necessidade de configurações adicionais. \cite{otg}
Exemplos de dispositivos compatíveis são controles da Playstation® e Xbox®. No caso dos controles mais antigos da Playstation® que não possuem cabos USB, é possível adquirir um adaptador como o da Figura ~\ref{f.adaptador}.

\begin{figure}[H]
	
	\begin{minipage}{.5\textwidth}{
			\centering
			\captionof{figure}{\small Cabo USB OTG}
			\includegraphics[height=4cm]{Imagens/otg.jpg}		
			\label{f.otg}	
			\legend{\small Fonte: Elaborada pelo autor.}
		}
	\end{minipage}
	\begin{minipage}{.5\textwidth}{
			\centering
			\captionof{figure}{\small Adaptador USB}
			\includegraphics[height=4cm]{Imagens/adaptador.jpg}		
			\label{f.adaptador}
			\legend{\small Fonte: Elaborada pelo autor.}	
		}
	\end{minipage}
	
\end{figure}

\subsection{Controle Bluetooth}

A tecnologia \textit{Bluetooth} está presente em diversos dispositivos ao redor do mundo, inclusive nos telefones celulares. Pode ser utilizada como forma de comunicação entre \textit{smartphones} e periféricos.

\begin{citacao}
	"Um dispositivo \textit{Bluetooth} utiliza ondas de rádio ao invés de cabos para se conectar a um celular ou computador. Um produto \textit{Bluetooth} como um fone de ouvido ou relógio, possui um pequeno chip de computador com rádio \textit{Bluetooth} e software que facilita a conexão. Quando dois dispositivos \textit{Bluetooth} querem trocar informações, eles precisam ser pareados." \cite[tradução nossa]{bluetooth}
\end{citacao}

Para este projeto, foram explorados dois dispositivos \textit{Bluetooth} que podem ser utilizados como dispositivos de controle do usuário com a aplicação em realidade virtual. O primeiro consiste em um teclado de computador que utiliza o \textit{Bluetooth} como forma de comunicação. Atualmente é possível encontrar este tipo de teclado que, além de se comunicar com o computador, também se conecta com dispositivos móveis sem que seja necessário a instalação de softwares específicos. 

O outro dispositivo utilizado foi um \textit{joystick} que utiliza a comunicação \textit{Bluetooth}. Atualmente encontra-se no mercado alguns controles com esta característica como o \textit{Wii Remote}, controle que acompanha o console Nintendo Wii® (Figura ~\ref{f.wiiremote}), e o controle da VR Box (Figura ~\ref{f.controlevrbox}) que acompanha o capacete de visualização da mesma empresa.

\begin{figure}[H]
	
	\begin{minipage}{.5\textwidth}{
			\centering
			\captionof{figure}{\small Wii Remote}
			\includegraphics[width=.7\linewidth]{Imagens/wiiremote.jpg}		
			\label{f.wiiremote}	
			\legend{\small Fonte: Elaborada pelo autor.}
		}
	\end{minipage}
	\begin{minipage}{.5\textwidth}{
			\centering
			\captionof{figure}{Controle VR Box}
			\includegraphics[width=.7\linewidth]{Imagens/controlevrbox.jpg}		
			\label{f.controlevrbox}
			\legend{\small Fonte: Elaborada pelo autor.}	
		}
	\end{minipage}
	
\end{figure}

Para este projeto, o \textit{joystick} utilizado foi o fornecido pela empresa VR Box pois o mesmo permite a conexão com dispositivos móveis diferentemente do \textit{Wii Remote} que não possui suporte para este tipo de conexão.

\section{Ferramentas de Desenvolvimento}
\label{s.ferramentas}

\subsection{Unity}
\label{ss.unity}
Tendo em vista que a aplicação foi feita em um dispositivo com sistema operacional Android, a ferramenta de desenvolvimento escolhida deve oferecer suporte para este tipo de dispositivo. A Google® oferece um conjunto de ferramentas denominado \textit{System Development Kit} (SDK) para desenvolvimento em RV. Os SDKs oferecidos contém todo o material necessário para a programação nas seguintes plataformas: Android, Unity, iOS e Unreal Engine 4. \cite{googledocumentacao} 

As opções contam com um projeto exemplo e materiais explicativos sobre o desenvolvimento de aplicações em realidade virtual. Além disso, a Google® oferece guias de boas práticas para a criação de aplicações em RV. 

Após análise, foi escolhido o SDK do Unity para o desenvolvimento da aplicação pois o mesmo oferece um ambiente gráfico intuitivo que não depende somente do código puro. Além disso, o Unity possibilita a exportação da aplicação para múltiplas plataformas (inclusive para dispositivos Android) de forma simples e rápida.

Segundo a empresa \citeonline{unitynumbers}, esta ferramenta é o principal software de desenvolvimento de jogos em escala global com 5 bilhões de jogos feitos em Unity baixados no 3º semestre de 2016 e 770 milhões de pessoas que jogam jogos feitos com a ferramenta. Na área da RV, é estimado que 90\% dos jogos desenvolvidos para o Samsung Gear VR e 53\% dos jogos desenvolvidos para o Oculus Rift foram feitos através do Unity.

O ambiente de desenvolvimento do Unity pode ser visualizado na Figura ~\ref{f.unity}, onde o menu do lado esquerdo contém os objetos presentes na cena, o centro contém a cena em si e o lado direito as propriedades dos objetos selecionados. Esta disposição dos elementos pode ser personalizada pelo usuário. Em cada elemento da cena, é possível adicionar \textit{scripts} que definirão as ações sobre os objetos, podendo ser decorrentes do \textit{input} do usuário e do estado da aplicação em geral. O Unity utiliza duas linguagens de programação para a criação de \textit{scripts}: C\# e UnityScript.  

\begin{figure}[H]
	\caption{\small Unity}
	\centering
	\includegraphics[height=6cm]{Imagens/unity.png}
	\label{f.unity}
	\legend{\small Fonte: Elaborada pelo autor.}
\end{figure}

A fim de criar uma imagem para cada olho e proporcionar a sensação de imersão, são utilizadas duas câmeras (uma para cada olho). De acordo com a documentação oficial do Unity, para mover ou girar a câmera, é preciso anexar as mesmas à um objeto. Desta forma, ao movimentar o objeto, as câmeras refletirão o movimento. 

\subsection{Integração Unity e Google VR}
\label{ss.unitygoogle}

Para facilitar o desenvolvimento de aplicações para o Google Cardboard e o Daydream, o Unity possui integração nativa com o Google VR. Para recursos adicionais, a Google® disponibiliza a Google VR SDK que requer a versão 5.2.1 ou superior do Unity e traz recursos como áudio espacial, suporte para o controle Daydream, ferramentas utilitárias e exemplos.  Segundo a \citeonline{googleunity}, a integração do Unity com o Google VR possibilita a localização da cabeça do usuário, renderização stereo, entre outros. 

\subsection{Capacetes de Visualização}
\label{ss.capacetes}

Para este projeto, foram utilizados dois capacetes de visualização: o Google Cardboard (Figura ~\ref{f.googlecardboard}) e o VR Box (Figura ~\ref{f.vrbox}). Ambos foram escolhidos devido ao baixo custo e por utilizarem um \textit{smartphone} como \textit{display}. 

\begin{figure}[H]
	
	\begin{minipage}{.5\textwidth}{
			\centering
			\captionof{figure}{\small Google Cardboard}
			\includegraphics[height=4cm]{Imagens/googlecardboard.png}		
			\label{f.googlecardboard}	
			\legend{\small Fonte: \citeonline{googlecardboard}.}
		}
	\end{minipage}
	\begin{minipage}{.5\textwidth}{
			\centering
			\captionof{figure}{\small VR Box}
			\includegraphics[height=4cm]{Imagens/vrbox.jpg}		
			\label{f.vrbox}
			\legend{\small Fonte: Elaborada pelo autor.}	
		}
	\end{minipage}
	
\end{figure}

Com o Google Cardboard todas as pessoas podem obter uma experiência de imersão em RV de forma simples e barata. É possível montar ou comprar um visualizador e obter a realidade virtual em \textit{smartphones} através de aplicações. \cite{googlecardboard}

O Google Cardboard 2.0 pode ser adquirido em diversos modelos como mostra a Figura ~\ref{f.modelos}. Apesar da primeira versão do Cardboard contar com um modelo para a montagem do visualizador, o modelo da segunda versão ainda não foi disponibilizado pela empresa, apesar de existir modelos de outras fontes.

\begin{figure}[H]
	\caption{\small Google Cardboard V2 Modelos}
	\centering
	\includegraphics[height=5cm]{Imagens/modelos.png}
	\label{f.modelos}
	\legend{\small Fonte: \citeonline{googlecardboard}.}
\end{figure}

Originalmente, o Google Cardboard não possui suporte para a fixação na cabeça do usuário, ou seja, o usuário terá que segurar o visualizador durante toda a experiência em RV, o que dificulta o uso de controles externos já que os mesmos deverão ser utilizados com somente uma das mãos do usuário. O visualizador comporta \textit{smartphones} com tela de 4.7 até 5.5 polegadas.

O visualizador VR Box vem acompanhado de um controle com comunicação \textit{Bluetooth} que já possui o suporte para cabeça. Diferentemente do Google Cardboard, o VR Box possui um compartimento ajustável para a inserção do \textit{smartphone} (Figura ~\ref{f.extensor}), possibilitando uma melhor fixação de \textit{smartphones} de diversos tamanhos (com telas de 4,7 até 6 polegadas) ao visualizador.

\begin{figure}[H]
	\caption{\small Compartimento para \textit{smartphone}}
	\centering
	\includegraphics[height=5cm]{Imagens/extensor.jpg}
	\label{f.extensor}
	\legend{\small Fonte: Elaborada pelo autor.}
\end{figure}



\subsection{Dispositivo Móvel}
\label{s.dispositivomovel}

Para se obter uma experiência completa em RV, é necessário que o dispositivo móvel possua giroscópio e acelerômetro. Ao adquirir o Google Cardboard é necessário observar as especificações do visualizador para saber quais tamanhos de telas são suportadas. 

Os testes previstos neste projeto foram realizados em um \textit{smartphone} da Motorola®, modelo Moto X XT1060. As especificações principais deste dispositivo podem ser visualizadas na Tabela ~\ref{t.motox}.

\begin{table}[H]	
	\caption{Especificações Técnicas Moto X} 
	\label{t.motox} 
	\centering
	\begin{tabular}{l|l}
		\textbf{\small Sistema operacional} & {\small Android 5.1 (Lollipop)} \\\hline
		
		\textbf{\small Dimensões} & {\small 129.3 x 65.3 x 10.4 mm (5.09 x 2.57 x 0.41 in)}  \\\hline	
		
		\textbf{\small Peso} & {\small 130 g (4.59 oz)}  \\\hline		 
		
		\textbf{\small Tela} & {\small AMOLED capacitiva touchscreen, 16M cores}  \\\hline  
		
		\textbf{\small Tamanho da tela} & {\small 4,7 polegadas} \\\hline
			
		\textbf{\small Resolução da tela} & {\small 720 x 1280 pixels} \\\hline
		
		\textbf{\small CPU} & {\small Dual-core 1.7 GHz Krait 300} \\\hline
		
		\textbf{\small GPU} & {\small Adreno 320} \\\hline
		
		\textbf{\small Memória} & {\small 16GB} \\\hline
		
		\textbf{\small Bluetooth} & {\small v4.0, A2DP, EDR, LE} \\\hline
		
		\textbf{\small USB} & {\small microUSB v2.0, USB Host} \\\hline	
		
		\textbf{\small Sensores} & {\small Acelerômetro, giroscópio, proximidade, bússola, barômetro, temperatura} \\\hline	
	\end{tabular}
	\legend{\small Fonte: \citeonline{motox}}	
\end{table}

\section{Aplicação ''Vença o Mosquito''}
\label{s.aplicacao}

\subsection{Descrição da Aplicação}
\label{ss.descricao}

Na aplicação desenvolvida, o usuário explora uma área que contém vários objetos. O objetivo do usuário é o de se movimentar no espaço criado por meio de um controle físico, procurando objetos específicos ao redor movimentando a cabeça e executando ações sobre os mesmos.

A fim de contextualizar o usuário no ambiente e definir os objetos que receberão ações, a aplicação trata o combate ao mosquito \textit{Aedes aegypti}, onde o usuário terá que eliminar os focos do mosquito no quintal de uma casa. 

O \citeonline{riodejaneiro} publicou em seu \textit{website} os lugares propícios para a reprodução do mosquito juntamente com as precauções que a população deve tomar para a eliminação do \textit{Aedes aegypti}. Os usuários da aplicação deverão realizar seis das quinze prevenções fornecidas pelo \textit{website} que são:

•	Coloque areia no prato dos vasos de plantas.

•	Mantenha o saco de lixo bem fechado e fora do alcance de animais até o recolhimento pelo serviço de limpeza urbana. Não jogue lixo em terrenos baldios.

•	Troque diariamente a água dos bebedouros de animais e aves e limpe-os com escova ou bucha.

•	Entregue seus pneus velhos ao serviço de limpeza urbana ou guarde-os sem água em local coberto e abrigados da chuva.

•	Guarde as garrafas vazias sempre de cabeça para baixo e de preferência em local coberto.

•	Limpe constantemente as calhas, remova tudo que possa impedir a passagem da água, a laje e a piscina de sua casa.

\subsection{Ações}
\label{ss.acoes}

O usuário conta com cinco ações dentro da aplicação: andar, agachar, selecionar, clicar e abrir o menu. Os caminhos possíveis que o usuário pode percorrer são determinados por objetos que servem como guias dentro da aplicação. Ao clicar nos guias, o usuário irá caminhar até os mesmos, podendo atingir seis posições no cenário como mostra a Figura ~\ref{f.posicoes}.

\begin{figure}[H]
	\caption{\small Posições possíveis na aplicação}
	\centering
	\includegraphics[scale=0.50]{Imagens/posicoes.jpg}
	\label{f.posicoes}
	\legend{\small Fonte: Elaborada pelo autor.}
\end{figure}

Os objetos que poderão sofrer ações possuem um menu que é acionado enquanto o usuário foca o objeto. Este menu mostrará qual ação é permitida para aquele determinado objeto e o usuário poderá executar a ação ao clicar no menu. A Tabela ~\ref{t.acoes} demonstra como cada controle executará as ações citadas acima. 

\begin{table}[H]	
	\caption{Ações dos dispositivos} 
	\label{t.acoes} 
	\centering
	\begin{tabular}{p{2.2cm}|p{2.2cm}|p{2.2cm}|p{2.2cm}|p{2.2cm}|p{2.2cm}}		
		\textbf{\small Dispositivo de controle} & \multicolumn{5}{c}{\textbf{\small Ações}}  \\ \hline \hline
		{} & {\small Andar} & {\small Agachar}  & {\small Selecionar} & {\small Clicar}  & {\small Abrir Menu}\\\hline \hline
		{\small Google Cardboard 2.0} & {\small Selecionar o objeto guia e efetuar o clique} & {\small Manter o botão pressionado} &{\small Olhar para o objeto} & {\small Pressionar e soltar o botão} & {\small Selecionar o botão do menu (localizado na porta da casa) e efetuar o clique}\\\hline		 
		{\small Controle PS2} & {\small Selecionar o objeto guia e efetuar o clique} & {\small Botão O} &{\small Olhar para o objeto} & {\small Botão X} & {\small Botão Select}\\\hline		 
		{\small Controle VRBox} & {\small Selecionar o objeto guia e efetuar o clique} & {\small O menor botão localizado na lateral frontal do controle} &{\small Olhar para o objeto} & {\small O maior botão localizado na lateral frontal do controle} & {\small Botão C}\\\hline  		 
		{\small Teclado} & {\small Selecionar o objeto guia e efetuar o clique} & {\small Botão “Down”} &{\small Olhar para o objeto} & {\small Botão “Space”} & {\small Botão “Escape”} 	\\\hline		
	\end{tabular}
	\legend{\small Fonte: Elaborada pelo autor}	
\end{table}

\chapter{Análise dos Controles}
\label{c.analise}

A fim de avaliar tecnologias sob o aspecto de uso, \citeonline{mcnamara} propuseram uma estrutura que se baseia em três aspectos: funcionalidade, experiência e usabilidade. A funcionalidade leva em consideração as características técnicas do dispositivo, experiência foca no relacionamento entre o usuário e a tecnologia e a usabilidade nas características de interação entre o usuário e o dispositivo. Os dispositivos utilizados para a realização da análise podem ser visualizados na Figura ~\ref{f.controles}. O controle via cabo utilizado acompanha o console de vídeo game \textit{Playstation 2} e é referenciado neste trabalho como PS2. 

\begin{figure}[H]
	\caption{\small Controles utilizados na análise}
	\centering
	\includegraphics[height=7cm]{Imagens/controles.jpg}
	\label{f.controles}
	\legend{\small Fonte: Elaborada pelo autor.}
\end{figure}

\section{Funcionalidade}
\label{funcionalidade}

Segundo \citeonline{mcnamara}, para avaliar a funcionalidade de um dispositivo pode-se analisar a performance, confiabilidade e durabilidade do mesmo. A quantidade de funções que um dispositivo oferece também deve ser considerada pois muitas opções de entrada podem inutilizar muitos comandos e poucas podem ser insuficientes. \citeonline{brown} afirma que a funcionalidade é um aspecto de interação que é relativamente independente do ambiente e do usuário. 

Quanto à performance, a característica analisada foi se o dispositivo ofereceu resposta rápida aos comandos do usuário, ou seja, se foram observados atrasos na comunicação entre o dispositivo e o celular. A facilidade de conexão do controle ao dispositivo móvel e a preservação desta conexão são características de confiabilidade. Por fim, a durabilidade levou em consideração o tipo de material de cada controle e se houve ou não falhas mecânicas na execução de comandos.

A análise foi feita com base em observações feitas pela autora onde cada controle foi utilizado para interagir com a aplicação desenvolvida. Cada controle foi analisado por em média dois minutos (tempo médio para destruir todos os focos de dengue apresentados na aplicação).

O Google Cardboard 2.0 mostrou ser o controle mais fácil de ser utilizado. Por não necessitar de conexão com o celular, a aplicação pôde ser inicializada rapidamente e não houve problemas técnicos durante a execução. Como o sistema de interação é o toque na tela, não foram observados nenhum atraso de resposta ou falhas mecânicas ao pressionar o botão. Em relação ao material, o Google Cardboard 2.0 é o que apresenta maior fragilidade em relação aos demais. Por ser quase inteiramente feito de papelão, o visualizador tem grandes chances de apresentar problemas em sua estrutura em um curto período de tempo. 

O segundo controle avaliado foi o PS2. Este possui uma qualidade de material superior em relação ao primeiro, porém, a conexão com o dispositivo não se manteve estável durante os testes. Isto ocorreu devido à problemas de compatibilidade com o Android 5.1 (\textit{Lollipop}), que não reconhece certos dispositivos através do cabo OTG. Testes realizados com um dispositivo Android 6.0 (\textit{Marshmallow}) não apresentou problemas de conexão. Para que o sistema utilizasse o controle como dispositivo de entrada, foi necessário instalar um aplicativo que funcionou como \textit{driver} para o controle utilizado, o que também dificultou a conexão inicial do dispositivo. O controle PS2 também não apresentou atrasos de resposta ou falhas mecânicas durante a análise.

Os controles \textit{Bluetooth} (\textit{joystick} do VR Box e teclado) apresentaram dificuldade na conexão inicial pois, diferentemente do cabo OTG, foi necessário conectar os dispositivos manualmente. Contudo, estes controles não apresentaram atrasos de resposta ou problemas de conexão. Além disso, a qualidade do material também mostrou ser superior ao Google Cardboard 2.0.

\section{Usabilidade}

A usabilidade, segundo a ISO 9241-11 \cite{iso9241} tem como objetivo definir usabilidade e explica como identificar a informação necessária para avaliação de usabilidade de um computador em termos de medidas de desempenho e satisfação do usuário, ou seja, mede o quanto um usuário específico pode utilizar um produto e atingir os seus objetivos com eficácia, eficiência e satisfação em um contexto específico de uso.

De acordo com \citeonline{brown}, eficácia descreve a habilidade do usuário em realizar uma tarefa com a tecnologia. Eficiência considera os recursos utilizados para realizar a tarefa, podem ser esforço mental, esforço físico ou tempo. A satisfação mede o quanto a interação impactou o usuário, devendo ser extraída somente através de respostas do usuário. 

Os questionários aplicados neste estudo tiveram como base a ISO 9241-9 e o trabalho de \citeonline{lewis} que é citado na ISO 9241-11 e oferece modelos de questionários para avaliação da satisfação de um produto. Por fim, o contexto de uso leva em consideração não somente o ambiente físico mas também as características individuais dos usuários. 

Para acessar a usabilidade oferecida por cada controle, 7 voluntários utilizaram a aplicação com os quatros diferentes controles e responderam a dois questionários elaborados por \citeonline{lewis}. O primeiro questionário é denominado \textit{"The After-Scenario Questionnaire (ASQ)"} (Apêndice ~\ref{a.apendice1}) e foi respondido para quatro cenários: andar, eliminar um foco de dengue, agachar e retornar ao menu. Este questionário possui apenas três questões que medem a facilidade para completar uma tarefa, tempo para completar a tarefa e se o suporte foi adequado. Como a análise é restrita aos controles e não à aplicação em si, o ASQ utilizado contém somente duas questões, já que problemas com o suporte não foram considerados. O segundo questionário, denominado \textit{"The Post-Study System Usability Questionnaire (PSSUQ)"} (Apêndice ~\ref{a.apendice2}) foi utilizado após a experiência completa da aplicação e foi respondido para cada controle. Como no primeiro questionário, questões que levavam em consideração a aplicação foram removidas. Ambos os questionários utilizam uma escala de 1 a 7 para cada pergunta sendo que 1 corresponde à "concorda totalmente" e 7 "discorda totalmente". No final, é feito a média aritmética das notas onde menores médias representam melhores avaliações.  

No total, 3 homens e 4 mulheres participaram da análise. A média de idade foi de 22,14, com idades variando entre 20 e 23 anos. Também foi questionado se os participantes possuíam experiência com algum dos controles apresentados e se já haviam utilizado aplicações em RV. Todos os participantes responderam que possuíam experiência com o teclado e o PS2, dois dos participantes disseram que também possuíam experiência com o Google Cardboard e apenas um deles já teve contato com todos os controles apresentados. Todos os voluntários já haviam entrado em contato com aplicações em RV.

Os resultados do ASQ podem ser visualizados na Tabela ~\ref{t.ASQ}.

\begin{table}[H]	
	\caption{Resultados após cada cenário (ASQ)} 
	\label{t.ASQ} 
	\centering
	\begin{tabular}{l|l}
	\textbf{\small Dispositivos } & \textbf{\small Média Aritmética} \\\hline
	
	{\small Google Cardboard 2.0} & {\small 1,75}  \\\hline	
	
	{\small Controle PS2} & {\small 1,43}  \\\hline		 
	
	{\small Controle VR Box} & {\small 1,43}  \\\hline  
	
	{\small Teclado} & {\small 1,79} \\\hline	
	\end{tabular}
	\legend{\small Fonte: Elaborada pelo autor}	
\end{table}

Os resultados apresentados mostraram que os controles PS2 e VR Box (\textit{joystick} que acompanha o visualizador) tiveram o melhor desempenho nos cenários propostos, apresentando maior facilidade e rapidez no uso de acordo com os participantes. A fim de saber se houve diferença significativa entre os resultados obtidos para cada controle, foi realizado a análise de variância ANOVA. O procedimento verifica se as médias de dois ou mais grupos são iguais através da comparação das variâncias dentro de cada grupo e entre os grupos \cite{minitab}. Esta comparação resulta no valor F que, se for menor do que o valor de F crítico (calculado pelo \textit{software} utilizado), então é possível concluir que não existe diferença entre as médias no ponto de vista estatístico. Este resultado é comprovado se o valor p é maior do que alfa (o valor padrão utilizado para alfa é 0.05). Neste experimento, cada grupo foi representado por um controle distinto.

Ao realizar a análise de variância ANOVA (Figura ~\ref{f.anovaasq}) por meio do \textit{software} Excel, foi constatado que não existe variância significativa entre os controles testados (F < F-crit). Porém, é importante observar que, ao aumentar a quantidade de cenários apresentados, alguns controles podem não apresentar suporte para ações extras. É o caso do Google Cardboard 2.0 que, por apresentar somente um botão, possui limitação para o número de ações. Já o teclado, por apresentar muitas opções de entrada, acaba tendo muito da sua capacidade inutilizada. 

\begin{figure}[H]
	\caption{\small Análise ANOVA para o ASQ.}
	\centering
	\includegraphics[scale=0.7]{Imagens/anovaasq.png}
	\label{f.anovaasq}
	\legend{\small Fonte: Elaborada pelo autor.}
\end{figure}

As avaliações de cada controle seguindo o questionário PSSUQ pode ser visualizado na Tabela ~\ref{t.PSSUQ}. 

\begin{table}[H]	
	\caption{Resultados após cada controle (PSSUQ)} 
	\label{t.PSSUQ} 
	\centering
	\begin{tabular}{l|l}
		\textbf{\small Dispositivos } & \textbf{\small Média Aritmética}\\\hline
		
		{\small Google Cardboard 2.0} & {\small 1,68} \\\hline	
		
		{\small Controle PS2} & {\small 1,33} \\\hline		 
		
		{\small Controle VR Box} & {\small 1,47	}  \\\hline  
		
		{\small Teclado} & {\small 2,04	 } \\\hline	
	\end{tabular}
	\legend{\small Fonte: Elaborada pelo autor}	
\end{table}

Os controles melhores avaliados pelos participantes, como no primeiro questionário, foram os controles PS2 e VR Box. A análise ANOVA (Figura ~\ref{f.anovapssuq}) demonstrou que não existe diferença significativa entre os controles (F < F-crit), porém, a variância de notas entre os controles foi maior no questionário PSSUQ. Apesar de todos os participantes terem experiência com os controles PS2 e teclado, é evidente a preferência ao PS2 para interagir com a aplicação. Observações realizadas durante o experimento constataram que os participantes tiveram dificuldade para segurar o teclado de forma confortável, o que poderia justificar a sua baixa avaliação. 

\begin{figure}[H]
	\caption{\small Análise ANOVA para o PSSUQ.}
	\centering
	\includegraphics[scale=0.7]{Imagens/anovapssuq.png}
	\label{f.anovapssuq}
	\legend{\small Fonte: Elaborada pelo autor.}
\end{figure}

\section{Experiência}
\label{experiencia}

De acordo com \citeonline[tradução nossa]{benyon}, a experiência “remete à todas as qualidades do sistema interativo que o fazem memorável, satisfatório e gratificante.” Tendo em mente os controles de interação, usuários podem obter uma experiência negativa se as características de funcionalidade forem insatisfatórias e se houver problemas para encontrar os botões corretos no controle, já que neste caso poderá ser necessária a remoção do capacete de visualização resultando em uma interrupção da experiência em RV. 

Para verificar a experiência oferecida por cada controle, foi perguntado aos usuários qual dos quatro controles resultou em uma melhor experiência. Adicionalmente, eventuais comentários feitos pelos participantes sobre os controles foram anotados e separados entre positivos e negativos. 

O resultado do experimento mostrou que os controles PS2 e VR Box foram os controles preferidos dos participantes com 3 votos cada, o voto remanescente foi para o Google Cardboard 2.0. O controle que menos agradou os participantes foi o teclado, que obteve 4 votos, seguido do Google Cardboard 2.0 que totalizou 3 votos.

Ao observar os participantes, foi possível notar que o cabo do PS2 era muito longo e houve uma preocupação por parte dos usuários para remover o cabo do caminho. No entanto, este detalhe não interferiu de forma significativa na preferência dos usuários pelo controle pois, segundo cometários dos participantes, o PS2 propiciou conforto durante os testes e, como todos os participantes declararam ter experiência utilizando o controle, nenhum apresentou dúvidas quanto ao posicionamento dos botões.

Quanto ao controle \textit{Bluetooth} VR Box, foram constatados opiniões diversas, todas em relação ao seu tamanho. Alguns participantes declararam que os botões ficavam muito juntos o que dificultava a interação, outros comentaram que o controle se encaixa na mão com facilidade.

O teclado foi o controle que recebeu mais opiniões negativas o que pode ser justificado pelo tamanho do dispositivo e pela quantidade de funções que ele oferece. Durante o experimento, os usuários mostraram ter dúvidas sobre qual seria a melhor maneira de segurar o teclado, algo que não ocorreu com os outros controles. Além disso, o número de entradas de dados disponível no controle é muito maior do que o exigido pela aplicação, o que deixou muitas teclas sem utilidade e aumentou as chances de erro do usuário ao utilizar a aplicação. 

Por fim, o Google Cardboard 2.0 recebeu elogios quanto à sua facilidade de uso. Porém, alguns usuários se queixaram que era preciso segurar o controle durante a experiência. Além disso, para retornar ao menu principal (um dos cenários apresentados) era necessário selecionar um botão localizado no ambiente e, para isso, o usuário precisou caminhar até o local ao invés de simplesmente acionar outro botão, resultando em um maior esforço em comparação com os outros controles.  
  

\chapter{Considerações Finais}
\label{c.conclusao}

Devido ao número de voluntários que participaram do estudo, não é possível generalizar os resultados obtidos. No entanto, as análises realizadas mostram uma tendência de preferência dos usuários. 

De acordo com a análise ANOVA realizada, a variância das avaliações não mostrou diferença significativa. Com esta informação, pode-se concluir que, para aplicações que envolvem poucas entradas de dados, os controles apresentados conseguem propiciar usabilidade satisfatória aos usuários. Na aplicação desenvolvida, o usuário pôde contar com três opções de entrada: clicar em um objeto selecionado, agachar e retornar ao menu principal. No caso do Google Cardboard 2.0, a última ação (retornar ao menu principal) não pôde ser efetuada através do controle, por isso, a aplicação contou com um botão virtual no ambiente, o que gerou alguns comentários negativos devido a necessidade do usuário de se locomover até o botão. 

O desenvolvimento deste projeto demonstrou que, se é intenção do usuário utilizar o Google Cardboard 2.0 como forma de entrada, é necessário preparar o ambiente adicionando elementos guias e botões adicionais, o que não seria necessário para os outros controles que possuem um número de botões elevado. Além disso, o usuário possui maior flexibilidade para se locomover no ambiente por meio dos botões direcionais físicos do controle do que através dos objetos guias no ambiente virtual (que limitam o número de posições dentro do cenário). Ao adicionar estes objetos, os botões direcionais dos controles tiveram sua utilidade anulada, deixando grande parte das opções de entrada de dados destes dispositivos inutilizada.

A partir destas informações, é recomendável que o desenvolvedor escolha um controle principal de acordo com as características da aplicação a ser desenvolvida. Para aplicações que envolvem somente a movimentação da cabeça e o clique para seleção, o Google Cardboard 2.0 é o controle mais apropriado, pois mostrou ser intuitivo e de fácil configuração. Aplicativos como \textit{VR Roller Coaster}, Netflix VR e Sistema solar VR disponíveis na Google Play \cite{googleplay}, são exemplos de aplicações desenvolvidas visando principalmente a utilização do Google Cardboard.

Já o teclado, apesar dos usuários terem experiência com este dispositivo, foi considerado o controle menos apropriado para aplicações em RV. Isto deveu-se principalmente pela quantidade excessiva de botões e pelo tamanho do dispositivo que trouxe certo desconforto aos usuários. O número de botões não interfere na usabilidade quando a quantidade de ações é pequena pois, por já possuírem experiência com o dispositivo, os usuários não apresentaram problemas para lembrar a localização de três botões. No entanto, ao aumentar o número de ações, botões no teclado podem ser facilmente confundidos e, como aplicações em RV não permitem que o usuário olhe para o controle a fim de encontrar o botão apropriado, seria necessário sair da experiência removendo o capacete de visualização. Quanto ao tamanho, comentários e observações feitos durante a análise demonstraram que o teclado foi o controle mais desconfortável comparado aos outros dispositivos apresentados. Uma característica positiva do teclado no entanto, é a conexão sem fio, que garante maior liberdade de movimentação. 

Os controles que garantiram maior satisfação dos usuários no quesito usabilidade foram os controles PS2 e VR Box. Estes últimos são conhecidos por serem utilizados em jogos e apresentam um número de botões satisfatório. Durante a análise, todos os usuários demonstraram ter experiência com o controle do PS2 e não apresentaram dificuldade para encontrar os botões corretos no dispositivo. Apesar de apenas um participante ter declarado possuir experiência com o VR Box, nenhum dos usuários apresentou problemas para encontrar os botões corretos. No entanto, houve comentários negativos como a distância entre os botões ser muito próxima e botões no centro do controle serem facilmente confundidos, o que seria um problema caso estes botões resultassem em ações diferentes (na aplicação apenas um botão do centro resultava em uma ação).

A grande diferença entre os controles preferidos, no entanto, foi a conexão com o \textit{smartphone}. Enquanto o VR Box é conectado ao celular através do \textit{Bluetooh}, o PS2 utilizou o cabo do próprio dispositivo, um adaptador e um cabo OTG para realizar a conexão, além de ser necessário um aplicativo para reconhecer o controle como um dispositivo de entrada devido a versão do Android utilizada. Apesar desta dificuldade, existem controles semelhantes à este que apresentam formas de conexão sem fio, como o controle do console XBOX (Figura ~\ref{f.xbox}) que realiza conexão Wifi e do console Playstation 4 (DUALSHOCK®4) que utiliza o \textit{Bluetooth} para se conectar com o dispositivo e pode ser visualizado na Figura ~\ref{f.ps4}.

\begin{figure}[H]
	
	\begin{minipage}{.5\textwidth}{
			\centering
			\captionof{figure}{\small Controle XBOX}
			\includegraphics[height=5cm]{Imagens/xbox.png}		
			\label{f.xbox}
			\legend{\small Fonte: \cite{xbox}}	
		}
	\end{minipage}
	\begin{minipage}{.5\textwidth}{
			\centering
			\captionof{figure}{\small DUALSHOCK®4}
			\includegraphics[height=5cm]{Imagens/dualshock4.jpg}		
			\label{f.ps4}	
			\legend{\small Fonte: \cite{ps4}}
		}
	\end{minipage}
\end{figure}


A utilidade destes controles no ambiente virtual pode ser reforçada devido ao fato de a empresa Oculus® oferecer o controle do XBOX juntamente com o seu óculos de realidade virtual (Rift). \cite{oculushome} 

A vantagem do VR Box em comparação com o PS2, PS4 e XBOX é a liberdade de movimentação do usuário pois o primeiro exige somente uma das mãos para se obter a experiência completa enquanto os últimos necessitam que o controle seja manejado com as duas mãos. Com a flexibilidade do VR Box, é possível adicionar sensores de movimentação no controle que captam a localização dos braços do usuário, possibilitando a criação de aplicações ainda mais imersivas, alterando a tecnologia para Virtualidade Aumentada. A empresa Oculus® além do controle XBOX, também oferece ao usuário a possibilidade de adquirir um segundo controle denominado Oculus Touch (Figura ~\ref{f.oculustouch}). Este apresenta a flexibilidade oferecida pelo VR Box e utiliza os sensores de movimento para o mapeamento dos braços do usuário na aplicação. 

\begin{figure}[H]
	\caption{\small Oculus Touch}
	\centering
	\includegraphics[height= 5cm]{Imagens/oculustouch.png}
	\label{f.oculustouch}
	\legend{\small Fonte: \cite{oculushome}}
\end{figure}

Estudos recentes demonstraram que é possível criar um controle como o Oculus Touch com um preço acessível. Os pesquisadores Fangwei Lee e Shuo Zhang desenvolveram o projeto RealTrigger \cite{realtrigger}. O projeto consiste em um controle feito a partir de papelão onde um \textit{QR code} é posicionado no mesmo de forma que a aplicação reconheça o código e mapeie este para a aplicação. O controle pode ser visualizado na Figura ~\ref{f.realtrigger}. 

\begin{figure}[H]
	\caption{\small RealTrigger}
	\centering
	\includegraphics[height= 7cm]{Imagens/realtrigger.png}
	\label{f.realtrigger}
	\legend{\small Fonte: \cite{realtrigger}}
\end{figure}

\begin{citacao}
	"Nós pensamos que experiências imersivas deveriam ser acessíveis para qualquer pessoa independente da situação econômica e foi por isso que desenvolvemos o RealTrigger: O controle de RV mais acessível que pode proporcionar uma experiência completa de interação em RV." \cite[tradução nossa]{realiteer}
\end{citacao}

Utilizar o RealTrigger juntamente com o VR Box ou outro \textit{joystick} poderia oferecer um controle imersivo acessível e com uma quantidade de ações e liberdade de movimento satisfatórias. 

De modo geral, a escolha do controle irá depender dos objetivos da aplicação. Aplicações que requerem muitas ações distintas do usuário devem contar com um controle externo como o VR Box ou controles utilizados em vídeo games como o PS4 e o XBOX. Controles que possuem conexão Wifi ou \textit{Bluetooth} são preferíveis por garantirem maior conforto e facilidade de conexão. Já aplicações que exigem apenas uma ação do usuário (normalmente o clique), podem contar com um controle mais simples como o Google Cardboard. 











% ---
% Capitulo com exemplos de comandos inseridos de arquivo externo 
% ---
% ---

% ----------------------------------------------------------
% ELEMENTOS PÓS-TEXTUAIS
% ----------------------------------------------------------
\postextual
% ----------------------------------------------------------


% ----------------------------------------------------------
% Referências bibliográficas
% ----------------------------------------------------------
\pagestyle{empty}

\bibliography{references} % o arquivo de bibliografia deve ser importando nessa linha sem o .bib
\begin{apendicesenv}
	\chapter{Introdução e questionário ASQ adaptado}
\label{a.apendice1}

\begin{flushleft}
Nome: \_\_\_\_\_\_\_\_\_\_\_\_\_\_\_\_\_\_\_\_\_\_\_\_\_\_\_\_\_\_\_\_ Idade:\_\_\_\_ Sexo(F|M):\_\_\_\_\_

Já possui experiência com o(s) controle(s) apresentados? Se sim, qual?:\_\_\_\_\_\_\_\_\_

Já teve contato com aplicações em Realidade Virtual?\_\_\_\_\_\_\_\_\_\_\_\_\_\_\_\_\_\_\_

Este questionário, que começa na página seguinte, lhe dá a oportunidade de me mostrar
suas reações com os controles que você utilizou. Suas respostas irão me ajudar a entender
quais aspectos do sistema que você tem preocupações e quais aspectos te satisfazem.
Para uma nota mais precisa, pense sobre todas as tarefas que você realizou com os
controles enquanto responde as perguntas.
Por favor, leia cada pergunta e indique o quão de acordo você está com a mesma
circulando um número na escala. Se uma pergunta não se aplica a você, não precisa
circular nenhum número.
Por favor, escreva comentários para elaborar as suas respostas.
Após completar o questionário, eu irei revê-lo para certificar que você entendeu todas as
suas respostas.

Obrigada!	

\newpage
Controle:\_\_\_\_\_\_\_\_\_\_\_\_\_\_\_\_\_\_

Cenário: Andar

Para cada uma das questões abaixo, circule a nota a sua escolha.

1. De modo geral eu estou satisfeito(a) com a facilidade para completar a tarefa.
\begin{figure}[H]	
	\centering
	\includegraphics[scale=0.6]{Imagens/questionarios.png}
	\label{f.questionarios}
\end{figure}

2. De modo geral eu estou satisfeito(a) com a quantidade de tempo que demorou para
completar a tarefa.
\begin{figure}[H]	
	\centering
	\includegraphics[scale=0.6]{Imagens/questionarios.png}
	\label{f.questionarios}
\end{figure}

Cenário: Eliminar um foco de dengue

Para cada uma das questões abaixo, circule a nota a sua escolha.

1. De modo geral eu estou satisfeito(a) com a facilidade para completar a tarefa.
\begin{figure}[H]	
	\centering
	\includegraphics[scale=0.6]{Imagens/questionarios.png}
	\label{f.questionarios}
\end{figure}
2. De modo geral eu estou satisfeito(a) com a quantidade de tempo que demorou para
completar a tarefa.
\begin{figure}[H]	
	\centering
	\includegraphics[scale=0.6]{Imagens/questionarios.png}
	\label{f.questionarios}
\end{figure}

Cenário: Agachar

Para cada uma das questões abaixo, circule a nota a sua escolha.
1. De modo geral eu estou satisfeito(a) com a facilidade para completar a tarefa.
\begin{figure}[H]	
	\centering
	\includegraphics[scale=0.6]{Imagens/questionarios.png}
	\label{f.questionarios}
\end{figure}

2. De modo geral eu estou satisfeito(a) com a quantidade de tempo que demorou para
completar a tarefa.
\begin{figure}[H]	
	\centering
	\includegraphics[scale=0.6]{Imagens/questionarios.png}
	\label{f.questionarios}
\end{figure}

Cenário: Retornar ao menu

Para cada uma das questões abaixo, circule a nota a sua escolha.

1. De modo geral eu estou satisfeito(a) com a facilidade para completar a tarefa.
\begin{figure}[H]	
	\centering
	\includegraphics[scale=0.6]{Imagens/questionarios.png}
	\label{f.questionarios}
\end{figure}

2. De modo geral eu estou satisfeito(a) com a quantidade de tempo que demorou para
completar a tarefa.
\begin{figure}[H]	
	\centering
	\includegraphics[scale=0.6]{Imagens/questionarios.png}
	\label{f.questionarios}
\end{figure}


\end{flushleft}
	\chapter{Questionário PSSUQ adaptado}
\label{a.apendice2}
\begin{flushleft}
Para cada uma das questões abaixo, circule a nota a sua escolha.

1. De modo geral eu estou satisfeito(a) com a facilidade utilizar o controle.
\begin{figure}[H]	
	\centering
	\includegraphics[scale=0.6]{Imagens/questionarios.png}
	\label{f.questionarios}
\end{figure}

2. Foi simples utilizar o controle. 
\begin{figure}[H]	
	\centering
	\includegraphics[scale=0.6]{Imagens/questionarios.png}
	\label{f.questionarios}
\end{figure}

3. Eu pude completar as tarefas com eficácia, ou seja, todas as tarefas foram cumpridas utilizando o controle.
\begin{figure}[H]	
	\centering
	\includegraphics[scale=0.6]{Imagens/questionarios.png}
	\label{f.questionarios}
\end{figure}

4. Eu pude completar as tarefas com rapidez utilizando o controle.
\begin{figure}[H]	
	\centering
	\includegraphics[scale=0.6]{Imagens/questionarios.png}
	\label{f.questionarios}
\end{figure}

5. Eu pude completar as tarefas com eficiência, ou seja, as tarefas foram realizadas corretamente utilizando o controle.
\begin{figure}[H]	
	\centering
	\includegraphics[scale=0.6]{Imagens/questionarios.png}
	\label{f.questionarios}
\end{figure}

6. Eu me senti confortável usando o controle.
\begin{figure}[H]	
	\centering
	\includegraphics[scale=0.6]{Imagens/questionarios.png}
	\label{f.questionarios}
\end{figure}


7. Foi fácil aprender a usar o controle.
\begin{figure}[H]	
	\centering
	\includegraphics[scale=0.6]{Imagens/questionarios.png}
	\label{f.questionarios}
\end{figure}

8. Eu acredito que poderia me tornar produtivo rapidamente utilizando o controle (Poderia efetuar rapidamente tarefas mais complicadas com o controle).
\begin{figure}[H]	
	\centering
	\includegraphics[scale=0.6]{Imagens/questionarios.png}
	\label{f.questionarios}
\end{figure}

9. Sempre que eu realizei um erro utilizando o controle eu pude recuperar do mesmo com rapidez e facilidade.
\begin{figure}[H]	
	\centering
	\includegraphics[scale=0.6]{Imagens/questionarios.png}
	\label{f.questionarios}
\end{figure}

10. O controle apresenta todas as funções e capacidade que eu esperava.
\begin{figure}[H]	
	\centering
	\includegraphics[scale=0.6]{Imagens/questionarios.png}
	\label{f.questionarios}
\end{figure}

11. De maneira geral, estou satisfeito(a) com o controle.
\begin{figure}[H]	
	\centering
	\includegraphics[scale=0.6]{Imagens/questionarios.png}
	\label{f.questionarios}
\end{figure}
\end{flushleft}
\end{apendicesenv}
% ----------------------------------------------------------
% Glossário
% ----------------------------------------------------------
%
% Consulte o manual da classe abntex2 para orientações sobre o glossário.
%
%\glossary

%---------------------------------------------------------------------
% INDICE REMISSIVO
%---------------------------------------------------------------------
\phantompart
\printindex
%---------------------------------------------------------------------

\end{document}
