\chapter{Problema}
\label{c.problema}

Apesar de o número de aplicações em RV para dispositivos móveis estar em crescimento, este número é pequeno em comparação com aplicações para smartphones sem a tecnologia de RV. Além disso, é atual os estudos de formas de interação com aplicações em RV que possam propiciar conforto, eficácia e conectividade adequada com smartphones. 

O Google Cardboard versão 1 propõe apenas duas formas de interação com o usuário: Movimentação da cabeça e um par de ímãs que quando utilizados representam um toque na tela, seu funcionamento é ilustrado na Figura 2. Já a versão dois não possui ímãs e utiliza o toque na tela como forma de interação como é ilustrado na Figura X. Atualmente, o suporte para a interação via ímã está sendo descontinuado e, por isso, as análises serão feitas utilizando o Google Cardboard versão 2.

\begin{figure}[h]
	\caption{\small Funcionamento do ímã}
	\centering
	\includegraphics[scale=0.7]{Imagens/cardboard.png}
	\label{f.cardboard1}
	\legend{\small Fonte: Elaborada pelo autor.}
\end{figure}

Possuir apenas duas formas de entrada limita as opções de interação do usuário com a aplicação. A empresa Oculus®, vende seus óculos de RV juntamente com um controle de vídeo game, o que garante maior flexibilidade na criação de aplicações em RV. No entanto, a diferença de preço entre o Google Cardboard e o Oculus Rift é alta. 



