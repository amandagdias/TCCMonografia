\chapter{Justificativa}
\label{c.justificativa}

A realidade virtual é um tema em expansão, objeto de pesquisa de grandes empresas como o Facebook® e a Google®. Logo, investigar formas de utilização desta tecnologia em dispositivos móveis como smartphones é contribuir com conhecimento para a área bem como auxiliar o trabalho de desenvolvedores. 

É interessante apresentar formas diferentes de interação com aplicações em RV e, ao mesmo tempo, obter uma experiência em realidade virtual acessível. O sistema operacional Android oferece bibliotecas que ajudam no tratamento de diversos meios de entradas como Bluetooth e via cabo. Além disso, o celular também possui sensores como acelerômetro e giroscópio que são utilizados para o desenvolvimento de aplicações em RV. Com o auxílio dos recursos mencionados, é possível realizar uma comparação entre diversos tipos de controles físicos, a fim de escolher o mais apropriado para ser utilizado em uma aplicação em RV. 

Por fim, o projeto pode ser realizado com materiais de baixo custo e de fácil acesso, podendo ser encontrados softwares completamente gratuitos.




